% ---
% Pacotes básicos
% ---
\usepackage{lmodern}      % Usa a fonte Latin Modern
\usepackage[T1]{fontenc}    % Selecao de codigos de fonte.
\usepackage[utf8]{inputenc}    % Codificacao do documento (conversão automática dos acentos)
\usepackage[table,usenames,dvipsnames,xcdraw]{xcolor}     % para colorir céluas de tabelas
\usepackage{lastpage}      % Usado pela Ficha catalográfica
\usepackage{indentfirst}    % Indenta o primeiro parágrafo de cada seção.
%\usepackage[usenames,dvipsnames]{color}    % Controle das cores
\usepackage{graphicx}      % Inclusão de gráficos
\usepackage{microtype}       % para melhorias de justificação
\usepackage{soulutf8}          % usado para grifar textos com o comando '\hl{}'
\usepackage[portuguese,obeyFinal,textsize=tiny]{todonotes}         % Pacotes para inserir anotações de coisas a fazer
\usepackage{paralist}          % pacote usado para enumerações "em linha"
\usepackage{url}               % pacote para adicionar URLs no texto
\usepackage{epstopdf}
\usepackage{array}             % usado para centralizar células de tabelas
\usepackage{longtable}
\usepackage{pdflscape}         % usado para landscape de página
\usepackage{adjustbox}
\usepackage{booktabs}
\usepackage{varwidth}%
\usepackage{underscore}        % Para adicionar underscores nas urls
\usepackage[final]{pdfpages}          % Para fazer o include de arquivos PDF
\usepackage{amsmath}            % Para fórmulas matemáticas
\usepackage{longtable}     %para tabelas com mais de uma página
\usepackage{capa-epusp-abntex2}
\usepackage{multirow}
% ---

% ---
% Pacotes de citações
% ---
\usepackage[brazilian,hyperpageref]{backref}   % Paginas com as citações na bibl
\usepackage[alf]{abntex2cite}  % Citações padrão ABNT
