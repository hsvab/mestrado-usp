% ---
% Capitulo Revisão de Literatura
% ---
\chapter{Revisão de Literatura}\label{chap:revisao-literatura}
% ---
\begin{citacao}
	\begin{flushright}  
\emph{``Todo enunciado - desde a breve réplica (monolexemática) até o romance ou o tratado científico - comporta um começo absoluto e um fim absoluto: antes de seu início, há os enunciados dos outros, depois de seu fim, há os enunciados-respostas dos outros [\ldots]. O locutor termina seu enunciado para passar a palavra ao outro ou para dar lugar à compreensão responsiva ativa do outro. O enunciado não é uma unidade convencional, mas uma unidade real, estritamente delimitada pela alternância dos sujeitos falantes'' (Bakhtin)}
	\end{flushright}
\end{citacao}

Este capítulo tem por objetivo clarificar conceitos considerando a evolução e as intersecções entre as concepções utilizadas, bem como dar um panorama geral de como as questões de gênero e de mobilidade vêm sendo tratadas sob a perspectiva do planejamento de transportes. 
Buscou-se, sempre que possível, apresentar aspectos ligados à realidade brasileira e quiçá paulistana, pois o escopo espacial de análise do presente trabalho é a Região Metropolitana de São Paulo (RMSP) (ver Figura \ref{fig:mapa-rmsp}), área coberta pela Pesquisas Origem e Destino (Pesquisas OD) do Metrô-SP. 

\begin{figure}[htb]%
    \caption{\label{fig:mapa-rmsp}Mapa dos municípios que compõem a região metropolitana em 2014, divididos por sub-regiões}%
    \begin{center}%
        \includegraphics[width=0.9\textwidth]{./imagens/Mapa-RMSP-subregions.png}%
    \end{center}%
    \fonte{Mapa elaborado por Marcos Elias Oliveira Júnior, segundo a Lei 1.139/2011 \cite{LEI1139}. Disponível em: \url{http://pt.wikipedia.org/wiki/Regi\%C3\%A3o_Metropolitana_de_S\%C3\%A3o_Paulo\#mediaviewer/File:Mapa-RMSP-subregions.svg} Acesso em 10 de novembro de 2014}
\end{figure}%

\section{Gênero}

Ao nascer, umas das primeiras atividades do ser humano é comunicar-se, o que inclui nominar para si o mundo que o cerca. Esse processo não se dá de maneira solitária, a nominação advém de uma interação social que visa compartilhar signos afim de efetivar a comunicação. Por isso, este capítulo visa estabelecer o que se deseja exprimir através de palavras-chave deste trabalho (gênero e mobilidade), considerando a visão de Bakthin, cujo trabalho, segundo \citeauthoronline{STELLA2005} (\citeyear{STELLA2005}), indicava ser necessário

\begin{citacao}
não somente a palavra, mas também a linguagem em geral, ser concebida e tratada de uma outra forma, levando-se em conta sua história, sua historicidade, ou seja, especialmente a linguagem em uso. Isso significa que, no pensamento bakthiniano, a palavra reposiciona-se em relação às concepções tradicionais, passando a ser encarada como um elemento concreto de feitura lógica.
\end{citacao}

Há um senso comum que confunde e funde, não por acaso, os conceitos de sexo e gênero, muito embora sejam distintos - distinção esta encontrada em maior ou menor grau de acordo com o idioma. Em inglês a palavra \emph{sex} tem sentido mais limitado, ligado à anatomia, e a palavra \emph{gender} tem sentido mais amplo, ligado à construção cultural da identidade. Em francês, a palavra \emph{séxe} e, em alemão, a palavra \emph{Geschlecht}, designam tanto diferenças físicas como psicológicas, sociais e culturais \cite{FRAISSE2001}. \citeauthoronline{MORAES1998} (\citeyear{MORAES1998}) reporta que, em francês, frequentemente utiliza-se \emph{rapports sociaux de séxe} ao invés de \emph{gendre} para se designar \emph{gênero}. Comparado com o termo inglês \emph{gender}, ``a palavra gênero, em português, é um substantivo masculino que designa uma classe que se divide em outras, que são chamadas espécies'', definição então retirada do Novo Dicionário Aurélio por \citeauthoronline{MORAES1998} (\citeyear{MORAES1998}, p.101).
Já hoje, em 2014, o Dicionário \citeauthoronline{AURELIOONLINE} (\citeyear{AURELIOONLINE}) comporta entre suas definições, aquela em que \emph{gênero} pode ser entendido como o ``conjunto de propriedades atribuídas social e culturalmente em relação ao sexo dos indivíduos''.
Porém, entre as definições mais gerais apresentadas pelo Dicionário \citeauthoronline{MICHAELIS2011} (\citeyear{MICHAELIS2011}), \emph{gênero} é definido da seguinte forma:

\begin{citacao}
s.m. (lat *\emph{generu}, por \emph{genus}) 1 Grupo de seres que têm iguais caracteres essenciais. 2 \emph{Lóg.} A classe que tem mais extensão e portanto menor compreensão que a espécie. 3 \emph{Biol.} Grupo morfológico intermediário entre a família e a espécie. 4 \emph{Gram.} Flexão pela qual se exprime o sexo real ou imaginário dos seres. 5 \emph{Gram.} Forma do adjetivo ou pronome com relação ao gênero dos nomes a que se refere. 6 Agrupamento de indivíduos que possuem caracteres comuns. 7 Espécie, casta, raça, variedade, sorte, categoria, estilo etc. 8 Qualidade, espécie, modo.
\end{citacao}

Percebe-se que o termo gênero designa um conceito em construção e consolidação, não apenas no Brasil, sendo necessário defini-lo sempre que o utilizarmos como denominação de categoria de análise \cite{MORAES1998}. Para isso, será feito um breve apanhado do surgimento e trajetória da palavra \emph{gender} ou \emph{gênero} nas pesquisas acadêmicas, inclusive no Brasil, bem como a evolução do conceito.

O conceito fundido e identitário de sexo e gênero, como se fossem sinônimos, pertence a uma visão binária de mundo que define as mulheres mais próximas da natureza, do trabalho reprodutivo, da passividade e do irracional e, em oposição, define os homens mas próximos à cultura, ao trabalho produtivo, à ação e à racionalidade \cite{HARAWAY2004}.
Estudiosas feministas, rejeitando o determinismo bio-sexual para a situação social das mulheres, precisavam desmontar a naturalização das diferenças entre homens e mulheres que vinculava suas relações sociais, políticas e econômicas a seu aparelho reprodutor \cite{PISCITELLI2009}. 
Para \citeauthoronline{HARAWAY2004} (\citeyear{HARAWAY2004}, p.218), as feministas lutaram ``para remover as mulheres da categoria da natureza e colocá-las na cultura como sujeitos sociais na história, construídas e auto-construtoras''. Dessa forma, a evolução do conceito de gênero mescla-se à história do feminismo.

A chamada ``primeira onda feminista'' ocorreu entre o final do século XIX e o início do século XX nos países hoje considerados desenvolvidos da Europa e da América do Norte. A principal bandeira reivindicava direitos iguais, compondo uma ideia de que deveria haver uma igualdade entre os sexos. Em decorrência dessa primeira movimentação, em diversos países, as mulheres conquistaram alguns direitos equivalentes aos dos homens, como o voto. Essa conquista do voto como um direito político caracterizou o movimento sufragista, que não pode ser confundido com o movimento feminista, embora seja parte dele.
A Finlândia foi o primeiro país a garantir direito a votar e ser votado(a) igualmente a mulheres e homens, em 1906, quando ainda era um Principado do Império Russo \cite{RAY1918}.
Na Inglaterra, em 1865, John Stuart Mill apresenta ao Parlamento um projeto de lei dando o voto às mulheres, que não foi aprovado. Somente em 1928, o voto feminino é autorizado nas mesmas condições às dos homens \cite{NELSON2004}.
Nos Estados Unidos, em 1920, foi aprovada a 19$^a$ Emenda%
\footnote{Fonte: \url{http://www.archives.gov/historical-docs/document.html?doc=13&title.raw=19th+Amendment+to+the+U.S.+Constitution:+Women\%27s+Right+to+Vote} Acesso em 02 de novembro de 2014.}, que proibia o estabelecimento de qualquer restrição ao voto (estadual e federal) baseada no sexo do(a) votante. 
 
Na década de 1930, Mead, uma antropóloga estadunidense, problematiza a fixitude dos conceitos \emph{feminilidade} e \emph{masculinidade} a partir de uma pesquisa comparativa entre três sociedades tribais na Nova Guiné  \cite{MEAD2000}. A pesquisadora conclui não haver um temperamento inato, universal que tenha origem biológica, ligada ao aparelho reprodutor. Ela observa que traços de caráter são aprendidos em sociedade, podendo, portanto, ser modificados e até desaprendidos. Ela deixa legado teórico que suporta a ideia de que existe uma construção cultural da diferença sexual.
%http://pt.scribd.com/doc/178229042/Resumo-Sexo-e-Temperamento-Margareth-Mead

Em 1949, a filósofa francesa Beauvoir lança a obra \emph{O Segundo Sexo}, considerado precursor da ``segunda onda feminista''\cite{PISCITELLI2009}. Ainda que \citeauthoronline{BEAUVOIR1967} (\citeyear{BEAUVOIR1967}) não cite o conceito de  ``papel social'' ou mesmo ``papel sexual'', ela enfatiza logo de início que o ``ser uma mulher'' é uma construção social:

\begin{citacao}
nenhum destino biológico, psíquico, econômico define a forma que a fêmea humana assume no seio da sociedade; é o conjunto da civilização que elabora esse produto intermediário entre o macho e o castrado que qualificam de feminino.
\cite[p.09]{BEAUVOIR1967}
\end{citacao}

Em sua obra, \citeauthoronline{BEAUVOIR1967} (\citeyear{BEAUVOIR1967}) tem por foco questionar a dominação masculina, sem deixar de questionar também a eficácia do movimento feminista, forjado até então no combate a essa dominação. 
Ela julgava ser possível esse combate ser bem sucedido se fossem combatidos elementos como: forma com que mulheres eram educadas; instituição de casamentos opressores; maternidade compulsória; vigência de um duplo padrão de moralidade sexual que permitia maior liberdade sexual somente aos homens; e falta de trabalhos dignos e bem remunerados que possibilitassem independência econômica às mulheres. 

Quase que concomitantemente, nos Estados Unidos, nasce um novo par de categorias de estudos, o sexo-gênero \cite{FRAISSE2001,STOLKE2004,HARAWAY2004}. A distinção entre as característica biológicas e as características sociais torna-se mais difundida, ou seja, na academia e na sociedade passa a ser considerada a noção de que posturas sociais de identidade masculina ou feminina não estabelecem relação biunívoca com o sexo anatômico.
A nominação dessa construção cultural pela palavra gênero ocorre em 1958, na Califórnia, quando foi empreendida uma pesquisa acerca da identidade de gênero no \emph{California Gender Identity Center}. Os resultados foram apresentados pelo psicanalista Robert Stoller em 1963, no Congresso de Pscicanálise de Estocolmo. Essa mesma pesquisa embasou a elaboração do primeiro volume de \emph{Sex and Gender} de \citeauthoronline{STOLLER1968} (\citeyear{STOLLER1968}). Essa obra expôs o quanto a relação sexo e gênero não é automática, nem estrita, discorrendo ainda sobre casos em que a anatomia da genitália não seria compatível com a identidade masculina ou feminina da pessoa. Assim, Stoller formula um conceito de gênero ligado à cultura, enquanto o conceito de sexo permanece ligado à morfologia corporal.

Em 1970 e 1980, o debate sobre esse par de categorias (sexo-gênero) toma espaço na comunidade acadêmica estadunidense. A antropóloga \citeauthoronline{RUBIN1975} (\citeyear{RUBIN1975})  introduz a categoria gênero no debate sobre opressões sociais sofridas pelas mulheres por meio do seu ensaio \emph{The Traffic in Women: Notes on the 'Political Economy' of Sex}. Nessa obra, \citeauthoronline{RUBIN1975} faz uma análise marxista sobreposta ao sistema sexo-gênero da qual depreende que no sistema de trocas capitalista, os homens estabelecem-se como vendedores e as mulheres são estabelecidas como mercadorias para serem trocadas.
Rubin dialoga com Lévi-Strauss que aponta ser o casamento o dispositivo mais importante de aliança entre as famílias, inexistente se não fosse pelo \emph{tabu do incesto} \cite{STRAUSS2010}. Para \apudonline{RUBIN1975}{PISCITELLI2009} esse tabu é precedido por outro, o da \emph{homossexualidade}. Isso porque, mediante a divisão sexual do trabalho%
\footnote{A expressão \emph{divisão sexual do trabalho} foi inicialmente utilizada por etnólogos para se referir à repartição das atividades entre homens e mulheres nas sociedades que estudavam \cite{KERGOAT2004}. Esta autora afirma ainda que ``a divisão sexual do trabalho é aquela decorrente das relações sociais de sexo'', o que será explorado mais adiante neste capítulo.} e ao tomar como a menor unidade de sobrevivência econômica a família, tem-se necessariamente um homem e uma mulher, numa relação heterossexual de dependência mútua. Rubin discute também o trabalho doméstico, dando visibilidade a um trabalho que muitas vezes viabiliza o sustento do trabalhador (geralmente homem) sem que seja remunerada (a mulher). Por fim, ela consegue articular teoricamente gênero e sexualidade de forma que o conceito de gênero constituído até então não reside apenas em identificação com um determinado sexo, mas pressupõe que o desejo sexual seja por indivíduo do sexo oposto. 
% \url{https://ensaiosdegenero.wordpress.com/tag/gayle-rubin/} 
% \url{http://ensaiosdegenero.wordpress.com/2012/04/16/o-conceito-de-genero-por-gayle-rubin-o-sistema-sexogenero/}

A distinção entre sexo e gênero foi extremamente útil às feministas acadêmicas, pois sinalizava um lastro teórico para embasar os estudos sobre a condição da mulher, muitas vezes inferiorizada por sua condição biológica inerente. Com isso, o questionamento à lógica binária de interpretação do mundo passou a ser menos frequente e incisivo \cite[p.218]{HARAWAY2004} e, porque não, superada em alguma medida. Conforme pode-se ver no trabalho de Rubin, o conceito de gênero foi além de separar dimensões culturais e biológicas de mulheres e homens. Cada vez mais o conceito de gênero passa a significar também a superação da leitura binária de mundo que só permite feminilidade ou masculinidade. Para \citeauthoronline{HEILBORN1992} (\citeyear{HEILBORN1992}, p.41):

\begin{citacao}
A categoria de gênero não deve ser acionada como um substituto de referência para homem ou mulher. Seu uso designa, ou deveria fazê-lo, a dimensão inerente de uma escolha cultural e de conteúdo relacional. Por outro lado, traz embutida a articulação desse código, que se apropria da articulação da diferença sexual tematizando-a em masculino e feminino, com outros níveis de significação dos universos.
\end{citacao}

Se a primeira onda do feminismo reivindicou direitos iguais, a segunda onda avançou e lutou pelo exercício igual dos direitos. Na primeira onda buscava-se provar que as diferenças entre o feminino e o masculino eram de origem social e não biológica. 
A afirmação não é abandonada na segunda onda, mas passa-se a buscar as origens de tais diferenças sócio-culturais. Nessa construção, segundo \citeauthoronline{PISCITELLI2009} (\citeyear{PISCITELLI2009}, p.133-134):

\begin{citacao}
A categoria ``mulher'' foi desenvolvida pelo feminismo da segunda onda em leituras segundo as quais a opressão das mulheres está além de questões de classe e raça, atingindo todas mulheres, inclusive as mulheres das classes altas e brancas. [...] O reconhecimento político das mulheres como coletividade ancora-se na ideia de que o que une as mulheres ultrapassa em muito as diferenças entre elas. Isso criava uma ``identidade'' entre elas.
\end{citacao}

Se essa uniformização entre as mulheres foi útil para forjar uma união na conquista por direitos, em meados da década de 1970 e início dos anos 1980, já era questionada. Feministas negras e mulheres de países subdesenvolvidos \cite{FURTADO2009} cada vez menos identificavam-se com o arcabouço teórico hegemônico e homogêneo apresentado por feministas dos países do ``norte rico'', inclusive por Rubin. Assim, a ``terceira onda feminista'' desdobra-se em feminismos diversos. Afinal, as mulheres negras contam com trajetória histórica diferente das mulheres brancas, grande parte das vezes tendo a escravidão e suas consequências como parte determinante da vida de sua ancestralidade \cite{HOOKS1990,CRENSHAW2002}. No caso de países subdesenvolvidos, como o Brasil, não cabe comparar \emph{ipsis literis} a trajetória das mulheres (mesmo brancas) brasileiras com as europeias. A título de ilustração, o estudo de \citeauthoronline{PINTO2004} (\citeyear{PINTO2004}) apresenta como as mulheres brasileiras são vistas como mais maternais, com vocação para a domesticidade e muito mais ``racializadas'' do que as portuguesas.

%Mas, segundo \citeauthoronline{WIZEMAN2001}(\citeyearonline{WIZEMAN2001}) os termos sexo e gênero não são sinônimos e, conforme definição adotada pelo Instituto de Medicina da \emph{National Academy of Sciences} o sexo é uma classificação ``de acordo com os órgãos reprodutores e funções [biológicas] atribuídas pelo complemento cromossômico''. Gênero, por sua vez, é a ``auto-representação de um pessoa como masculino ou feminino, ou como a pessoa é percebida por instituições sociais com base na apresentação de gênero do indivíduo''.

Oferecendo alguma resposta a essas demandas por interseccionalidade%
\footnote{Interseccionalidade ou abordagem interseccional, segundo \citeauthoronline{CRENSHAW2002} (\citeyear{CRENSHAW2002}, p.177) ``trata especificamente da forma pela qual o racismo, o patriarcalismo, a opressão de classe e outros sistemas discriminatórios criam desigualdades básicas que estruturam as posições relativas de mulheres, raças, etnias, classes e outras''.} em 1986, a historiadora pós-estruturalista Joan Scott publica seu artigo \emph{Gender: A Useful Category of Historical Analysis} em que faz uma leitura crítica da utilização do termo \emph{gênero} como categoria de análise e relaciona necessariamente esta categoria a outras como classe e raça, pois demonstra ser o gênero necessariamente imbricado a relações hierarquizadas de poder:

\begin{citacao}
a oposição binária e o processo social das relações de gênero tornam-se,
ambos, partes do sentido do próprio poder. Colocar em questão ou mudar um aspecto ameaça o sistema por inteiro. Se as significações de gênero e de poder se constroem reciprocamente, como é que as coisas mudam? [\ldots] o gênero tem que ser redefinido e reestruturado em conjunção com uma visão de igualdade política e social que inclui não só o sexo, mas também, a classe e a raça. \cite[p.1073,1075]{SCOTT1986}
\end{citacao}

%Assim como Scott, a filósofa estadunidense Judith Butler também tem influência foucaultiana e é pós-estruturalista. Em sua obra \emph{Gender Trouble: Feminism and the Subversion of Identity} publicada em 1990 Butler questiona a coerência entre sexo (biológico), gênero (construção cultural) e desejo (sexual). Para ela, existe uma regra tácita  heterossexual socialmente aceita como correta, estimulada, e que exige uma determinada coerência na tríade sexo-gênero-desejo. A partir dessa foram de ler o gênero, articulado ao desejo sexual, é que pessoas transgênero passam a ter algum arcabouço teórico que lhes abarque. \citeauthoronline{BUTLER1999} (\citeyear{BUTLER1999}) descreve a performatividade, logo, para ela, o gênero seria um ato intencional, performativo e que gera significados.

É então necessário olhar a construção das identidades de gênero à luz das relações de poder e olhar brevemente como se deu a evolução dos direitos, especialmente na sociedade brasileira. As mulheres no Brasil escravocrata dispunham de uma grande imobilidade geográfica e mesmo as mulheres das classes dominantes raramente saíam às ruas e, quando o faziam, nunca estavam desacompanhadas \cite{SAFFIOTI2013}. Mulheres e homens de então desfrutavam de maneira assimétrica do direito de ir e vir.

Na campo dos direitos políticos, o movimento sufragista das brasileiras não teve tanta capilaridade nem foi um movimento de massas como nos Estados Unidos, Inglaterra ou Rússia. Ele teve início na década de 1910, quando o Partido Republicano Feminino é fundado no Rio de Janeiro com o objetivo de instaurar o debate acerca do voto feminino%
\footnote{Bertha Lutz, filha do cientista Adolfo Lutz, licenciou-se em Ciências Naturais na Sorbonne de Paris e, ao retornar ao Brasil, funda a Federação Brasileira pelo Progresso Feminino, em 1919, que leva adiante a luta pelo sufrágio feminino \cite{PINSKY2003}. A primeira cidade a autorizar o voto feminino em eleições foi Mossoró (RN), em 1928. Em nível nacional, Getúlio Vargas autoriza em 1931 o voto feminino apenas às mulheres solteiras, viúvas com renda própria ou casadas com a autorização do marido.}.
A igualdade de condições de voto entre homens e mulheres se concretiza em 1932, pelo Decreto nº 21.076 que autoriza o voto a qualquer cidadã ou cidadão com idade superior a 21 anos. A eleição de 1933 foi a primeira em que mulheres puderam participar do pleito, votando e sendo votadas, como Carlota Pereira Queiroz, a primeira deputada brasileira, que participou da Assembleia Nacional Constituinte entre 1934 e 1935 \cite{TABAK1989}.

Embora o direito ao voto tenha sido emblemático, a ideia de desfrutar de \emph{direitos iguais} na sociedade, mulheres e homens, tratava também de outros direitos como o acesso à educação e poder ter posse de bens - por muito tempo, de acordo com a lei, só homens podiam ser proprietários de casas, por exemplo \cite{PISCITELLI2009}. Subjacente a esses questionamentos das mulheres tecia-se o conceito de ``papel social'', bastante difundido a partir da década de 1930. Para \citeauthoronline{PISCITELLI2009} (\citeyear{PISCITELLI2009}, p.127), a teoria dos papeis sociais buscava:

\begin{citacao}
compreender os fatores que influenciam o comportamento humano. A ideia é que os indivíduos ocupam posições na sociedade, desempenhando papeis de filho, de estudante, de avô. [...] A ideia de posições ocupadas no desempenho dos papeis faz referência a categorias de pessoas que são reconhecidas coletivamente. Um dos atributos que podem servir de base para a definição dessas categorias é a idade. [...] Outro desses atributos pode ser o sexo. Nesse caso, homens e mulheres desempenham papeis culturalmente construídos: os papeis sexuais.
\end{citacao}

Essa busca por um leque de direitos não foi um movimento só das mulheres, mas um movimento de luta por cidadania%
\footnote{A cidadania, para \citeauthoronline{CARVALHO2002} (\citeyear{CARVALHO2002}), é entendida como o exercício pleno de três direitos: direitos civis, direitos sociais e direitos políticos. Os civis são aqueles considerados direitos fundamentais, como o direito à vida, à liberdade, à propriedade, à igualdade perante a lei. Eles garantem a vida em sociedade e dependem da existência de uma justiça independente, eficiente, barata e acessível a todos. Os políticos se referem à participação do cidadão no governo da sociedade. Seu exercício é limitado a uma parcela da população definida por idade, por exemplo, e consiste na capacidade de fazer demonstrações políticas, de organizar partidos, de votar, de ser votado. Por fim, os sociais são aqueles que garantem a participação na riqueza coletiva e se baseia na ideia de justiça social. Incluem os direitos à educação, ao trabalho, ao salário justo, à saúde, à aposentadoria.}. Sob o ponto de vista de gênero e cidadania, \citeauthoronline{BRITO2001} (\citeyear{BRITO2001}) relembra que o conceito clássico de cidadania, o grego, excluía mulheres e escravos. Ela pontua que ao longo da história as identidades de homens e mulheres foram construídas pressupondo uma dicotomia entre o âmbito público e o privado. \citeauthoronline{BLAY2001} (\citeyear{BLAY2001}) relata que até os anos 1960/1970 era um fator negativo para a mulher participar da vida pública.
A partir de 1970, com o movimento feminista passa a haver críticas e questionamentos quanto à natureza, à separação e à natural atribuição dessas esfera a um determinado sexo. Assim, elabora-se uma perspectiva de análise a partir do gênero, e não do sexo biológico, pois o conceito de gênero também compreende as dimensões social e política do termo.

A partir dos anos 1990 o uso da categoria gênero tornou-se mais frequente no Brasil e cada vez mais influenciado pelas diversas escolas de psicanálise para explicar a produção e a reprodução da identidade de gênero do sujeito.
A psicanalista brasileira \citeauthoronline{KEHL1998} (\citeyear{KEHL1998}) embora não tenha como central esse debate, participa dele e em sua obra \emph{Deslocamentos do Feminino}, ao invés de apartar sexo de gênero, assinala que gênero é um conceito que inclui a dimensão biológica do sexo, não sem somar-lhe atributos que a cultura provê.
A partir de Scott é cada vez mais corrente incorporar a dimensão da política e do poder na composição do conceito de gênero, conforme explicita \citeauthoronline{MORAES1998} (\citeyear{MORAES1998}, p.100):

\begin{citacao}
A expressão relações de gênero, tal como vem sendo utilizada no campo das ciências sociais, designa, primordialmente, a perspectiva culturalista em que as categorias diferenciais de sexo não implicam o reconhecimento de uma essência masculina ou feminina, de caráter abstrato e universal, mas, diferentemente, apontam para a ordem cultural como modeladora de homens e mulheres. Em outra palavras, o que chamamos de homem e mulher não é o produto da sexualidade biológica, mas sim de relações sociais baseadas em distintas estruturas de poder.
\end{citacao}

Essas relações de poder incidem tanto sobre as relações que se desdobram no espaço público, quanto as do espaço privado. No espaço público, ou não-doméstico, tem relevância para o presente estudo o mercado de trabalho. Ao longo do tempo, a urbanização e a industrialização levaram à ampliação da classe média e ao crescimento do consumo no Brasil. As mulheres entraram também neste processo, embora a maior parte das trabalhadoras tenha sido absorvida, ao menos inicialmente, no setor de serviços e com enorme concentração nos empregos domésticos, de menor rendimento. Constata-se assim que existe aqui uma divisão sexual do trabalho \cite{KERGOAT2004}, que tem por características a destinação prioritária dos homens à esfera produtiva e das mulheres à esfera reprodutiva. Essa forma de divisão pauta-se em dois princípios: o da separação e o da hierarquização. O princípio da separação explicita a ideia de que há ``trabalhos de mulheres'' e ``trabalhos de homens'' enquanto o princípio da hierarquização indica existir uma diferença de valoração entre o trabalho do homem (produtivo, mais valioso) e o da mulher (reprodutivo, menos valioso). \citeauthoronline{BLAY2001} (\citeyear{BLAY2001}, p.84) fala sobre a mulher brasileira:

\begin{citacao}
Até a década de 1960 a história, quando focalizava a mulher, atinha-se às supostas atividades femininas fundamentais, isto é, de um ser apêndice da família. A historiografia simplesmente ignorava a participação feminina no mercado de trabalho, a enorme freqüência com que sustentavam economicamente a si e aos seus.
\end{citacao}

Ao longo do século XX, observou-se aumento de mulheres na população economicamente ativa brasileira%
\footnote{A população economicamente ativa (PEA) é obtida pela soma da população ocupada e desocupada com 16 anos ou mais de idade. ``População ocupada'' compreende as pessoas que, num determinado período de referência, trabalharam ou tinham trabalho mas não trabalharam (por exemplo, pessoas em férias). ``População desocupada'' compreende as pessoas que não tinham trabalho, num determinado período de referência, mas estavam dispostas a trabalhar, e que, para isso, tomaram alguma providência efetiva nos últimos 30 dias (consultando pessoas, jornais, etc.). Fonte: IBGE - disponível em \url{http://www.ibge.gov.br/apps/snig/v1/?loc=0,355030&cat=118,119,1,2,-2,-3&ind=87} Acesso em 21 de novembro de 2014} 
(ver Gráfico \ref{graf:evolucao-pea}). 
%e a maior frequência feminina em empregos de jornadas menores (ver Gráfico \ref{graf:percent-jornadas})
O fenômeno permanece no século XXI de acordo com estudo da \citeauthoronline{ABRAMO2010} (\citeyear{ABRAMO2010}):
(i) observa-se que de 2001 a 2010 manteve-se a preponderância feminina em ocupações que demandam de 20 a 40 horas semanais;
(ii) para homens, manteve-se o predomínio histórico de jornada superior a 40 horas semanais;
(iii) o ingresso das mulheres no mercado de trabalho não alterou drasticamente o papel delas na família e, portanto, nas atividades ligadas às tarefas domésticas. Isto é, apesar de muitas mulheres terem entrado no mercado de trabalho algumas décadas atrás, elas ainda são responsáveis pela maior parte do trabalho doméstico. No Gráfico \ref{graf:jornadas-completas} é possível constatar não apenas essa divisão sexual do trabalho - produtivo, no mercado, e reprodutivo, no lar - mas também que as jornadas totais que acabam ficando a cargo da mulher são maiores. Os homens acumulam uma jornada de cerca de 50 horas por semana, as mulheres, 57 horas semanais.

\begin{grafico}[htb]%
    \caption{\label{graf:evolucao-pea}Percentual de indivíduos que fazem parte da PEA, por sexo, no Brasil, entre 1950 e 2010}%
    \begin{center}%
        \includegraphics[width=1.10\textwidth]{./imagens/evolucao-pea1.png}%
    \end{center}%
    \fonte{Adaptado de \cite{ALVES2013}}
\end{grafico}%

%\begin{grafico}[htb]%
%    \caption{\label{graf:percent-jornadas}Percentual de trabalhadores(as) com jornadas de trabalho semanal acima de 44 horas e 48 horas e abaixo de 35 horas, por sexo, no Brasil, em 2008}%
%    \begin{center}%
%        \includegraphics[width=0.9\textwidth]{./imagens/jornada-muler2003.png}%
%    \end{center}%
%    \fonte{\apud[p.01]{PNAD2008}{OIT2008}}
%\end{grafico}%

\begin{grafico}[htb]%
    \caption{\label{graf:jornadas-completas}Jornadas Médias para o Mercado de Trabalho e para Reprodução Social, por sexo, raça/cor e região geográfica, no Brasil, em 2003}%
    \begin{center}%
        \includegraphics[width=0.9\textwidth]{./imagens/jornadas-totais.png}%
    \end{center}%
    \fonte{PNAD (2003 apud \citeauthoronline{SOARES2003}, \citeyear{SOARES2003})}
    \nota{Segundo \citeauthoronline{DEAR1981} (\citeyear{DEAR1981}), o espaço de reprodução é onde a recuperação da força de trabalho ocorre sendo a residência o local principal a ser considerado; e o espaço de produção é onde o processo de acumulação do capital ocorre, ou seja, no que se denomina mercado de trabalho (indústria, comércio e serviços no geral).}
\end{grafico}%

%Mais um parágrafo sobre a divisão sexual do trabalho, dos papeis sociais - citar hirata aqui!

Esse é o ponto em que a atuação no espaço público e a no espaço privado vincula-se.
A mulher passa a poder desempenhar atividades antes tidas como ``masculinas'', porém sem ser desonerada de desempenhar as atividades tidas como ``femininas'', pois  ainda ``persistem nichos onde vigora uma imagem feminina vinculada à maternidade e ao cuidado da família, à saúde da prole'' \cite[p.94]{BLAY2001}. 
Assim, a ampliação do leque de papéis sociais que a mulher desempenha impacta as relações de poder dentro do ambiente doméstico, dentro da família. Isso molda as necessidades, interesses, atividades e padrão de viagens dos integrantes da família, a partir das identidades de gênero constituídas, forjadas pelos comportamentos de indivíduos e da relação de poder estabelecida entre eles.

\clearpage
\section{Mobilidade Urbana}

A palavra mobilidade, de acordo com o Dicionário \citeonline{MICHAELIS2011},
significa ``(i) propriedade do que é móvel ou do que obedece às leis do movimento;
(ii) deslocamento de indivíduos, grupos ou elementos culturais no espaço social;
(iii) movimento comunicado por uma força qualquer;
(iv) falta de estabilidade, de firmeza ou inconstância''.
Tal definição reflete toda uma gama de conceitos relacionados a movimento e/ou deslocamento, o que na área de transportes relaciona-se imediatamente a viagens.
No Brasil, há cerca de 100 anos, a maior parte das viagens de pessoas era feita a pé%
\footnote{Os primeiros carros foram montados em São Paulo pela Ford na década de 1910. Fonte: \url{http://www.carroantigo.com/portugues/conteudo/curio_hist_carro_brasileiro.htm} Acesso em 25 de outubro de 2014} 
e, quando muito, usava-se tração animal (cavalo ou boi), especialmente para cargas. 
Isso incorria em baixas velocidades de deslocamento e, assim, grande parte das pessoas acabavam por desenvolver suas atividades, por toda vida, nas proximidades de onde nasceram. Neste último século o cenário mudou bastante, as mais diversas tecnologias se desenvolveram, os rendimentos aumentaram, a mobilidade aumentou \cite[p.06]{METZ2012}, e como exemplos icônicos dessa maior mobilidade figuram a utilização do carro e do avião.

O conceito de mobilidade pode englobar muitos outros e se desdobrar em uma grande diversidade de temas.
Há quem o aborde ligando-o ao turismo \cite{ENLOE1989,FROHLICK2008}, enquanto outros \cite{CHANT1992,SILVEY2000} abordam-no sob a perspectiva dos movimentos migratórios entre países ou dentro de uma mesma nação. Outro uso do termo é ligado ao intercâmbio de estudantes e pesquisadores de diferentes instituições de origem – o que dá origem à expressão \emph{mobilidade acadêmica} \cite{ENDERS1998,TREMBLAY2005,HOFFMAN2008}. Ademais, há um olhar sobre a mobilidade em que as condições geodemográficas são elementos de contorno, delimitando assim as áreas da mobilidade rural e urbana.
%Embora pareçam óbvias à primeira vista – porque em alguma medida vividas – as diferenças entre rural e urbano são bem menos claras quando olhadas mais de perto. No Brasil as distinções nascem de critérios político-administrativos, originados em decreto de 1938%
%\footnote{Decreto-Lei nº 311, de 2 de Março de 1938 disponível em: \url{http://www2.camara.leg.br/legin/fed/declei/1930-1939/decreto-lei-311-2-marco-1938-351501-publicacaooriginal-1-pe.html} Acesso em 06 de novembro de 2014}
%de Getúlio Vargas e, até hoje, no Brasil, baseia-se em critérios políticos administrativos. Segundo o IBGE, ``como situação urbana consideram-se as áreas correspondentes às cidades (sedes municipais), às vilas (sedes distritais) ou às áreas urbanas isoladas''%
%\footnote{Conceitos adotados no censo do IBGE disponíveis em: \url{http://www.ibge.gov.br/home/estatistica/populacao/contagem/conceitos.shtm}}. Como rural, classifica-se tudo o que não se configure urbano. Trata-se de definição legal (jurídica) ocorrida frequentemente no campo da política e passível de críticas, como a de \cite{GRABOIS2001} que aponta que tal classificação não considera as diferentes funções dos aglomerados como critério. A questão da definição do que é rural vai além da abordagem teórica e tem como pano de fundo as diferenças de tributação entre as áreas rural e urbana. Como saída a essa arbitrariedade que fica a cargo dos poderes municipais, \citeauthoronline{VEIGA2002} (\citeyear{VEIGA2002}) elenca três critérios que entende importantes a se considerar nesse tipo de classificação: (i) população total do município, (ii) densidade demográfica e (iii) localização.
A delimitação espacial deste trabalho é a RMSP%
\footnote{Os 39 municípios que compõem a RMSP são agrupados em 6 regiões de acordo com Lei Complementar estadual nº 1.139, de 16 de junho de 2011. Na região central está São Paulo. Na região Sudoeste encontram-se 8 municípios, a saber, Juquitiba, São Lourenço da Serra, Embu-Guaçu, Itapecerica da Serra, Embu, Tabão da Serra, Cotia e Vargem Grande Paulista. Na Região Oeste encontram-se 7 municípios, a saber, Pirapora do Bom Jesus, Santa de Parnaíba, Barueri, Jandira, Itapevi, Carapicuiba, Osasco. Na região Norte encontram-se 5 municípios, a saber, Cajamar, Caieira, Franco da Rocha, Francisco Morato, Mairiporã. Na região Leste encontram-se 11 municípios, a saber, Santa Isabel, Arujá, Guarulhos, Itaquaqueceteuba, Guararema, Poá, Suzano, Ferraz de Vasconcelos, Mogi das Cruzes, Biritiba Mirim, Salesópolis. Na região Sudeste encontram-se 7 municípios, a saber, Santo André, São Bernardo do Campo, São Caetano do Sul, Diadema, Mauá, Ribeirão Pires, Rio Grande da Serra.}; onde todos municípios englobados contam, atualmente, com áreas consideradas urbanas
\footnote{Conceito de urbano adotado no censo do IBGE disponível em: \url{http://www.ibge.gov.br/home/estatistica/populacao/contagem/conceitos.shtm}}. 
%Muito embora se vá seguir as classificações oficiais do IBGE e do Metrô-SP, entende-se como salutar esta breve discussão sobre o significado do que é ser área urbana ou rural, pois ``o espaço rural tem passado recentemente por um conjunto de mudanças com significativo impacto sobre suas funções e conteúdo social'' \cite[p.96]{MARQUES2002}. Deixa-se a ressalva de que mesmo na área de enfoque, urbana, encontram-se atividades agrícolas (comumente tidas como rurais), afinal, são cada vez mais imprecisos os limites entre um e outro \cite{MINGIONE1987}. Um outro fenômeno que merece alguma atenção, dada a natureza deste trabalho, é o êxodo rural seletivo que vem sendo constatado por alguns pesquisadores. \cite{RAUBER2010} constata que no Rio Grande do Sul a emigração do campo é desigual em gênero e em idade: mulheres e jovens migram mais, homens e idosos são os que permanecem no campo, nas atividades rurais. Fenômeno semelhante é constatado em Santa Catarina%
%\footnote{Êxodo seletivo é retratado em Santa Catarina pelo documentário ``Celibato no Campo'' de Cassemiro Vitorino e Ilka Goldschmidt, 2013, disponível em \url{http://www2.camara.leg.br/camaranoticias/tv/materias/OLHARES/440520-CELIBATO-NO-CAMPO.html} Acesso em 15 de outubro de 2014.},
%e em alguns países europeus%
%\footnote{Relatório do Parlamento Europeu em 2003 apontava que somente 37\% da mão-de-obra rural da União Europeia era de mulheres. Disponível em \url{http://www.europarl.europa.eu/sides/getDoc.do?type=REPORT&reference=A5-2003-0230&format=XML&language=PT} Acesso em  16 de outubro de 2014}. Ou seja, existe diferença relacional entre a mobilidade feminina e a masculina expressa, por exemplo, nos deslocamentos campo-cidade.
E o foco de análise da presente dissertação concentra-se nos deslocamentos intra-urbanos, expressos amplamente pelo conceito de mobilidade urbana. 

%Se ``o estudo dos problemas urbanos é indissociável da relação campo-cidade'' \cite[p.154]{FREITAG2007} e por isso, independente da época estudada, é preciso tê-la em mente; especificamente 
A mobilidade urbana é um elemento fundamental para que seja possível garantir aos habitantes de uma cidade acesso aos bens que lhes oferece \cite{IEMA2010}. Cabe aqui diferenciar o que seja mobilidade do que seja acessibilidade. A mobilidade exprime a capacidade de se deslocar no espaço, ``refere-se à habilidade de mover-se entre dois diferentes locais de atividade'' \cite[p.04]{HANSON1995a}.
Sobre a acessibilidade e sua ligação com a mobilidade, \citeauthoronline{HANSON1995a} afirma ainda:

\begin{citacao}
A acessibilidade refere-se ao número de oportunidades [\ldots] disponíveis dentro de uma determinada distância ou tempo de viagem. [\ldots] Conforme as distâncias entre os locais de atividades se tornam maiores [\ldots] a acessibilidade passa a depender cada vez mais da mobilidade, particularmente daquela relacionada aos veículos particulares. \cite[p.04]{HANSON1995a}
\end{citacao}

Dessa maneira, vinculando a mobilidade à motorização, \citeauthoronline{HANSON1995a} afirma que é possível promover acessibilidade sem incrementar a mobilidade (1995, p.05), afinal, ter toda uma sorte de serviços próximos à residência daria a possibilidade de ir à padaria, ao mercado, à igreja, à escola, à livraria, etc. a pé. A autora ainda se aprofunda na questão da acessibilidade ao classificá-la como (i) de pessoas ou (ii) de lugares. Trata-se apenas de diferentes referenciais, a acessibilidade de uma pessoa indica o quão fácil ou difícil é para ela chegar a determinado local; a acessibilidade de um lugar mostra o quão fácil ou difícil pode ser alcançá-lo. Essas expressões ``fácil'' e ``difícil'', entretanto, são formas muita genéricas para caracterizar o que se deseja exprimir. Problema para o qual ela apresenta como resposta uma medida de acessibilidade $A_{i}$, onde $A_{i}$ é o conjunto das oportunidades $O_{i}$ ponderadas pelas distâncias $d_{i,j}$ da residência da pessoa \emph{i}:

\begin{equation}\label{eq:acessibilidade}
A_{i} = \sum_{i}^{} O_{i} d_{i,j}^ {-b}
\end{equation}

Caso considere-se ao invés de \emph{i} indivíduos, \emph{i} zonas, a Equação \ref{eq:acessibilidade} referir-se-á à zona \emph{i} de análise. Como este é um modelo bastante simplificado, indica apena um potencial de acessibilidade e tem suas limitações, como por exemplo, desconsiderando a dimensão temporal dos deslocamentos.
\citeauthoronline{VASCONCELLOS2012} (\citeyear{VASCONCELLOS2012}), por sua vez, pontua a questão temporal em sua definição de acessibilidade:

\begin{citacao}
medida pela quantidade e/ou diversidade de destinos que a pessoa consegue alcançar, por certa forma de transporte, em determinado tempo. Quanto maior for esta quantidade, maior é a acessibilidade, ou seja, mais oportunidades as pessoas terão para realizar atividades desejadas ou necessárias. \cite[p.42]{VASCONCELLOS2012}
\end{citacao}

O mesmo autor, em obra anterior, indica como forma de mensurar a acessibilidade, a soma dos tempos: (i) de deslocamento até o meio de transporte; (ii) tempo de espera, caso exista; (iii) tempos(s) dentro do(s) meio(s) de transporte; (iv) tempo de transferência entre diferentes meios de transportes, caso exista; (v) tempo após saída do meio de transporte até atingir o destino final. Destes tempos, ele classifica os itens (i) e (v) como \textbf{microacessibilidade}, ou seja, itens que referem-se ``à facilidade relativa de ter acesso aos veículos ou destinos desejados (por exemplo, condições de estacionamento ou acesso ao ponto de ônibus)'' (\citeyear{VASCONCELLOS2001}, p.91). A \textbf{macroacessibilidade} é definida por ele como a:

\begin{citacao}
facilidade relativa de atravessar o espaço e atingir construções e equipamentos urbanos desejados. Ela reflete a variedade de destinos que podem ser alcançados e, consequentemente, o arco de possibilidades de relações sociais, econômicas, políticas e culturais, dos habitantes do local. \cite[p.91]{VASCONCELLOS2001}
\end{citacao}

Isto posto, a acessibilidade pode ser entendida como a capacidade de se chegar onde se deseja e, para sua mensuração pode-se usar a distância e/ou o tempo. Uma das formas de reunir essas duas dimensões é através do prisma espaço-tempo. Na Figura \ref{fig:prisma1} pode-se observar o diagrama do tempo em função da distância\footnote{Para ser efetivamente um prisma, tal gráfico deveria ser em três dimensões: na base x e y representariam os deslocamentos no plano, e na altura, z, teríamos o tempo.} de um casal hipotético com filho pequeno. Supôs-se que ambos trabalhem das 8 às 18 horas, que seja a mulher a levar e buscar a criança na escola e que haja um carro na família.
A Figura \ref{fig:prisma1} descreve uma situação em que o pai fica com o carro da família. Assim, a mulher leva a criança na escola a pé%
\footnote{Considerou-se que a velocidade média ao caminhar seja de uma velocidade média de 5km/h. Fonte: \url{http://www.anpet.org.br/xxviiianpet/anais/documents/AC301.pdf} Acesso em 03 de dezembro de 2014.} e segue para o trabalho de ônibus enquanto seu marido segue para o trabalho de carro, ambos no sentido bairro-centro%
\footnote{Para esta simulação ilustrativa foram usadas velocidades do pico da manhã de 16km/h para ônibus e de 21km/h para carros, de acordo com dados da CET. Fonte: \url{http://www.cetsp.com.br/media/228073/2007\%20\%20volumes\%20e\%20velocidades.pdf} Acesso em 06 de dezembro de 2014.}.
Na volta, ele retorna de carro diretamente para a residência e ela sai do trabalho de ônibus, apanha a criança na escola, para depois irem a pé para casa; todos no sentido centro-bairro.%
\footnote{Para esta simulação ilustrativa foram usadas velocidades do pico da tarde de 11km/h para ônibus e de 19km/h para carros, de acordo com dados da CET. Fonte: \url{http://www.cetsp.com.br/media/228073/2007\%20\%20volumes\%20e\%20velocidades.pdf} Acesso em 03 de dezembro de 2014.}
São acessíveis à mulher as oportunidades contidas na área de hachura vermelha. A área azul indica o diferencial de acessibilidade que o homem tem neste caso. Ou seja, a cidade possível no que diz respeito a oportunidades de trabalho, escola, lazer, etc. ao alcance dessa mulher é quase a metade daquela desse homem.

\begin{figure}[htb]%
    \caption{\label{fig:prisma1}Prisma Espaço Tempo}%
    \begin{center}%
        \includegraphics[width=0.9\textwidth]{./imagens/DIAGRAMA-ESPACO-TEMPO-final.eps}%
    \end{center}%
    \fonte{Elaboração própria}
\end{figure}%

A situação narrada é hipotética, mas é ilustrativa de alguns fatores que restringem as liberdades de movimento, segundo \apudonline{HAGERSTRAND1970}{HANSON1995a}:
\begin{compactitem}[]
\item (i) limitações devido ao fato de que não se pode estar em dois lugares ao mesmo tempo e que certas tarefas precisam ser feitas usando um determinado modo de transporte (por algum motivo pode não haver outras opções);
\item (ii) necessidade de encaixar os compromissos de uma pessoa com os de outra(s) pessoa(s), como por exemplo, levar filho(a)(s) à escola, acompanhar idoso(a)(s) ao médico ou almoçar com amigo(a)(s);
\item (iii) restrições devido à autoridade social, política e/ou legal no acesso a algum lugar - pode haver regras explícitas (ou implícitas), por exemplo, que impeçam as pessoas de andarem à noite sozinhas num determinado local.
\end{compactitem}

Geralmente, as pessoas não dispõem de várias alternativas modais ou porque não têm recursos (ou para ter um carro ou para morar em área bem servida de transporte público) ou porque se sentem inseguras utilizando algum modo específico (medo de ser assaltado(a) no carro, de ser atropelado(a) de bicicleta, de ser assediado(a) no transporte público, entre outros). Não são todos(as) que podem arranjar horários flexíveis de trabalho e/ou de estudos para conciliar as outras atividades da forma mais eficiente, por exemplo, evitando os deslocamentos nos horários de pico. Enfim, as restrições incidem diferentemente nos grupos sociais criando assimetrias no acesso às oportunidades. Essa iniquidade desdobra-se no espaço: na distribuição desigual de empregos nas cidades, na variação de preço do solo urbano, na densidade heterogênea de serviços públicos oferecidos nos diversos bairros e mesmo na ocupação do espaço de circulação (ver Figura \ref{fig:equidade}).

\clearpage

\begin{figure}[htb]%
    \caption{\label{fig:equidade}Quantidade de espaço viário requerido para transportar 60 pessoas por ônibus, bicicleta e carro.}%
    \begin{center}%
        \includegraphics[width=0.80\textwidth]{./imagens/equidade-sqn.png}%
    \end{center}%
    \fonte{Foto de \emph{Cycling Promotion Fund}, disponível em \url{http://www.bhtrans.pbh.gov.br/portal/
page/portal/portalpublico/Temas/ObservatorioMobilidade/FiquePorDentro/ObsMobBH\%20A\%
20cidade\%20com\%20menos\%20carros} - acesso em 22 de novembro de 2014.}
\end{figure}%

Em 1991, o estrato com os 20\% de menor renda da população perfazia 9\% das milhas viajadas nos Estados Unidos (por carro e ônibus), ao passo que o estrato com os 20\% de maior renda concentrava 32\% do total de milhas\apud{CAMERON1994}{HANSON1995a}.
\citeauthoronline{VASCONCELLOS2001} (\citeyear{VASCONCELLOS2001}) apresenta o consumo de espaço, por modo de transporte e renda da RMSP em 1987 e 1997 (ver Tabela \ref{tab:consumoespaco}). Observa-se que na RMSP os grupos de maior renda tendem a consumir mais espaço de circulação, o que levanta a questão do quão (in)justo é esse cenário, principalmente no Brasil, onde o sistema de tributação que custeia a infra-estrutura pública urbana é regressivo%
\footnote{O Brasil conta com um sistema de tributação regressivo, ou seja, aquele em que a retirada é proporcionalmente maior das pessoas com menor capacidade de contribuir \cite{GRECO2005}.}.

Segundo \apudonline{URRY2004}{UTENG2008}, tendo em vista as iniquidades urbanas de acesso, estruturadas socialmente, há cinco ``mobilidades'' bastante interdependentes:
\begin{compactitem}[]
\item (i) viagem corpórea das pessoas por motivo de trabalho, lazer, etc.;
\item (ii) movimento físico de objetos (cargas);
\item (iii) viagem imaginativa a lugares por meio de imagens (fotos ou televisão);
\item (iv) viagem virtual mediante uso da internet; 
\item (v) viagem comunicativa através de mensagens trocadas entre pessoas (cartas, mensagens de celular, telefone).
\end{compactitem}

\clearpage

\begin{table}[htb]
    \IBGEtab{%\renewcommand{\arraystretch}{1.5}%%\ABNTEXfontereduzida%
	    \renewcommand{\arraystretch}{1.5}
        \caption{Consumo dinâmico de espaço por modo e renda na RMSP}
		\label{tab:consumoespaco}
    }{%
	    \begin{tabular}{P{3.50cm} P{3.5cm} P{3.5cm} P{3.5cm}}
            \toprule
	           \headerCell{Renda familiar mensal (1987)} &
		       \headerCell{Espaço dinâmico (km*$m^2$/dia/pessoa) (1987)} &
   	           \headerCell{Renda familiar mensal (1997)} &
		       \headerCell{Espaço dinâmico (km*$m^2$/dia/pessoa) (1997)}
	           \\
		    \midrule \midrule
		        0 a 240&
		        7,6&
		        0 a 250&		        
		        9,2\\
		    \midrule
		        241 a 480&
		        13,4&
		        251 a 500&		        
		        14,6\\
		    \midrule
		        481 a 900&
		        25,1&
		        501 a 1000&		        
		        23,7\\
		    \midrule
		        901 a 1800&
		        42,2&
		        1001 a 1800&		        
		        36,7\\
		    \midrule
		        1801 ou mais&
		        74,8&
		        de 1801 a 3600&		        
		        56,2\\		        
		    \midrule
		        -&
		        -&
		        3601 ou mais&		        
		        81,9\\		        
		    \bottomrule
		\end{tabular}
    }{%
		\fonte{Adaptado de \cite[p.181;196]{VASCONCELLOS2001}}
		\nota{Considerando consumo médio de $1,0m^2/pessoa$ em transporte público ($30m^2$ de área de ônibus para 30 passageiros em média) e $6,6m^2/pessoa$ em transporte privado ($10m^2$ de área de carro para 1,5 passageiros em média).}		
		}
\end{table}

Não é possível considerar então a mobilidade do indivíduo, isolando-o do seu contexto social, econômico, político e cultural; muito pelo contrário, só é possível entendê-la se considerarmos os ambientes em que o indivíduo se ancora: doméstico, familiar e social \cite{HANSON2010}. Portanto, acrescido do significado de urbano definido pelo IBGE, trabalha-se aqui com o conceito de mobilidade relacionado ao item (i) de \apudonline{URRY2004}{UTENG2008}, mais detalhado no artigo \emph{Gender and mobility: new approaches for informing sustainability} de \citeauthoronline{HANSON2010}, que emprega o termo \textbf{mobilidade} para designar:

\begin{citacao}
o movimento de pessoas de um lugar para outro lugar no decorrer da vida cotidiana [\ldots] [sendo a] principal preocupação com as viagens pessoais que compõem a rotina diária de atividades como o trabalho (remunerado e não remunerado), lazer, socialização e compras.
\cite[p.7]{HANSON2010}
\end{citacao} 

% Seção de Sustentabiidade na Revisão de Literatura
% ---
%\clearpage
\section{Sustentabilidades}

Por definição de sustentabilidade encontram-se nos dicionários descrições bastante simples e amplas, que podem ser resumidas como ``a qualidade de ser sustentável'' \cite{MICHAELIS2014}, o que posterga a dúvida para a questão: o que é ser sustentável? Segundo \citeauthoronline{BLACK2010} (\citeyear{BLACK2010}), é aquilo que pode ser mantido ou que dure. É evidente que tal durabilidade não é eterna, mas por um determinado período. A ideia da permanência leva a crer que tal período seja longo, que relacione-se à perspectiva de longo prazo, mas de quão longo se trata, é uma indefinição, até hoje. Hoje, inclusive, é cada vez mais frequente o uso do termo sustentável como modificador ao invés de sustentabilidade como um conceito fechado em si. Aqui, exploraremos o desenvolvimento  sustentável e o transporte sustentável como as ``sustentabilidades'' de interesse.

As primeiras preocupações e dicotomizações entre desenvolvimento e meio ambiente remontam ao fim da década de 1960, com o Clube de Roma%
\footnote{O Clube de Roma fora fundado em 1968 por Aurelio Peccei e Alexander King e consistia num grupo de pessoas ilustres (empresários, líderes religiosos, políticos, entre outros) que se reuniam para discutir assuntos ligados à política, economia e, também, meio ambiente. Para saber mais: \url{http://www.clubofrome.org/} Acesso em 06 de novembro de 2014.}, que será o berço da obra \emph{The Limits to Growth}. Este livro, publicado em 1972, problematiza pela primeira vez a questão do crescimento exponencial \emph{versus} a finitude dos recursos disponíveis e também se propõe a simular e tentar prever as consequências da interação antrópica com sistemas não-antrópicos \cite{MEADOWS1972}. Nesse mesmo ano, ocorre a Conferência sobre o Ambiente Humano das Nações Unidas em Estocolmo.

Em 1983, as Nações Unidas (ONU) fundam a Comissão Mundial sobre Meio Ambiente e Desenvolvimento (WCED), composta por 19 delegados de 18 países, com a missão de produzirem um estudo sobre desenvolvimento em escala global, considerando aspectos como sustentabilidade e meio ambiente num perspectiva de longo prazo. Assim, a expressão ``desenvolvimento sustentável'' aparece pela primeira vez em \citeyear{WCED1987}, no relatório \emph{Our Common Future} da WCED, também conhecido como \emph{Brundtland Report}%
\footnote{Gro Harlem Brundtland era o Primeiro Ministro da Noruega e foi quem comandou a Comissão Mundial sobre Meio Ambiente e Desenvolvimento (WCED). Fonte: \url{http://www.un-documents.net/our-common-future.pdf} Acesso em 06 de novembro de 2014.}, onde é apresentado o clássico conceito:

\begin{citacao}
desenvolvimento sustentável é aquele que satisfaz as necessidades do presente sem comprometer a capacidade das gerações futuras satisfazerem as suas próprias necessidades. Este conceito contém em si outros dois conceitos-chave: o de ``necessidades'', em particular as necessidades essenciais dos pobres do mundo, às quais deve ser dada prioridade absoluta; e a ideia de limitações impostas pelo estágio tecnológico e de organização social sobre a capacidade do meio ambiente de satisfazer as necessidades presentes e futuras. 
\cite[p.41]{WCED1987}
\end{citacao}   

Essa definição consolidou-se na Conferência das Nações Unidas sobre Meio Ambiente e Desenvolvimento ocorrida em 1992 no Rio de Janeiro, também conhecida como ECO-92. Em quase vinte anos, o conceito popularizou-se, ganhou robustez e também ficaram mais nítidas suas limitações de implementação. Um relatório de balanço da ONU publicado em 2010 \cite{ONU2010} indica haver convergência conceitual de que o desenvolvimento sustentável está alicerçado sobre três pilares: desenvolvimento econômico, equidade social e proteção ambiental. O mesmo documento reconhece, porém, que apesar de visionário e integrador, o conceito tem se mostrado de difícil implementação pelos países e pouco tem sido abraçado em sua completude pelas políticas das diversas nações. 

Embora o desenvolvimento sustentável pretenda englobar os três pilares, o desenvolvimento é frequentemente sinônimo de desenvolvimento econômico e a sustentabilidade fica muitas vezes compartimentada à questão ambiental \cite{ONU2010}. Um dos caminhos que vem sendo construído para tentar superar essa dificuldade é tornar o conceito menos difuso e mais palpável por meio de indicadores \cite{CHAMBERS2000,BOULANGER2008,BARRETT2010,FORTES2012} e metas (de preferência quantitativas) a serem atingidas num determinado prazo \cite{ONU2010,ONU2014}.

Outra saída, não excludente como esta recém apresentada, é adicionar outros pilares na conceituação do que seja desenvolvimento sustentável, como faz 
\citeauthoronline{BANISTER2005} (\citeyear{BANISTER2005}). Ele elenca outros dois fatores como fundamentais: (i) participação e (ii) governança. A dimensão da participação refere-se a envolver todas as pessoas interessadas e envolvidas no processo, a saber, indivíduos, empresas, indústrias e governos. Argumenta que criar excluídos do processo torna muito mais difícil desenvolver as estratégias necessária de mudança. A dimensão da governança incide diretamente no processo de tomada de decisão, logo, significa mudanças nas estruturas organizacionais para que sejam facilitadas decisões intersetoriais.

\citeauthoronline{BANISTER2005} (\citeyear{BANISTER2005}) destaca cinco lições aprendidas desde o \emph{Brundtland Report} a partir das experiências de sucessos e fracassos: (i) as medidas que implicam redução de consumo  ou impactam estilo de vida devem começar modestamente; (ii) desestímulo às emissões de dióxido de carbono (CO$_2$) deve contar com mecanismos fiscais; (iii) deve haver incentivos fortes em pesquisa e desenvolvimento em ciência e tecnologia na temática das mudanças climáticas; (iv) embora todos países devam contribuir para a diminuição de emissões de carbono, a liderança cabe às nações mais ricas; (v) é preciso ação imediata e incerteza não é uma boa razão para inação ou atitudes fracas.
Ao se falar em sustentabilidades, fala-se necessariamente de mudanças de paradigma, profundas, e que podem até mesmo ser inatingíveis \cite{GLASBY2002}, embora possam ser perseguidas. \citeauthoronline{ROGERS2000} (\citeyear{ROGERS2000}) aponta que essa mudança deve necessariamente compreender as ``relações entre cidadãos, serviços, políticas de transporte e geração de energia, bem como seu impacto total no meio ambiente local e numa esfera geográfica mais ampla'' e aponta ainda as cidades, pensadas como organismos vivos, cujo metabolismo é bastante linear, precisando tornar-se mais circular (ver figura \ref{fig:cidade-metabolismos}).

\begin{figure}[htb]%
    \caption{\label{fig:cidade-metabolismos}Cidades de metabolismo linear e circular}%
    \begin{center}%
        \includegraphics[width=0.80\textwidth]{./imagens/richard-linear-circular.jpg}%
    \end{center}%
    \fonte{\cite[p.31]{ROGERS2000}}
\end{figure}%

Do ponto de vista econômico, a obra \emph{Limits to Growth} fez escola e trouxe à baila a hipótese de que o crescimento econômico teria um teto, função dos recursos (naturais) disponíveis. Contudo, há estudiosos que se opõem a isso, como \citeauthoronline{KRUGMAN2014} (\citeyear{KRUGMAN2014}) em seu recente artigo \emph{Slow Steaming and the Supposed Limits to Growth}. O economista argumenta que é possível manter o crescimento econômico real (do PIB) e ainda assim reduzir a emissão de gases do efeito estufa. Para chegar a essa conclusão ele apresenta uma demonstração - bastante simplista - que considera o consumo de energia dos navios em função de suas velocidades e conclui ser possível manter um caudal econômico constante e, concomitantemente, diminuir o consumo de energia do sistema.

Entre as principais preocupações no âmbito ambiental estão a diminuição das reservas de petróleo, aquecimento global por conta da emissão de gases do efeito estufa, poluição (atmosférica, sonora e hídrica) e presença de chuva ácida. Diversos estudos indicam que existe correlação entre a ocorrência de diversos tipos de doenças (cardiorrespiratórias, câncer, entre outras) e a exposição a alguns poluentes presentes na atmosfera \cite{WHO2000,WHO2006,BRUNEKREEF2012,MIRANDA2012}. Em São Paulo, \citeauthoronline{GOUVEIA2006} (\citeyear{GOUVEIA2006}) observaram associação estatisticamente significante entre o aumento no nível de poluentes na atmosfera e o aumento de hospitalizações por causas diversas, em todos grupos etários estudados. Entre 1971 e 2001, as emissões de CO$_2$, indicado como o principal gás responsável pelo efeito estufa, aumentaram cerca de 60\% e a parcela cuja origem são os sistemas de transporte também aumentou de 19,3\% para 28,9\% \cite{BANISTER2005}.

Sob o prisma da equidade social, o acesso equânime a oportunidades de educação, trabalho, saúde e lazer é um dos pontos centrais. A equidade, associada à ideia do ``ser justo'', inevitavelmente referir-se-á à distribuição social de custos e benefícios, bem como em que grau essa distribuição é considerada adequada e que corrobore para a promoção da justiça \cite{LITMAN2006}. Aqui também o transporte tem papel estruturador já que pode ser o elemento que provê ou barra o acesso às oportunidades. \citeauthoronline{SANCHEZ2003} (\citeyear{SANCHEZ2003}) já apontavam que equidade seria um dos temas estratégicos nas políticas de transportes. As megalópoles latino-americanas são, por vezes, cidades ``partidas'' \cite{VENTURA2001} entre a ``legal'' e a ``real'' \cite{ALVA1997}%
\footnote{Estima-se que cerca de 40\% ou mais da população possui moradia em condição irregular \cite{FREITAG2007}.},
onde as vias de circulação frequentemente são cicatrizes no tecido urbano - por exemplo, em São Paulo, o ``Minhocão'' \cite{ABASCAL2010}, os monotrilhos \cite{ROLNIK2010} ou mesmo uma rua na favela de Paraisópolis (ver Figura \ref{fig:paraisopolis}), em São Paulo.

\begin{figure}[htb]%
    \caption{\label{fig:paraisopolis}Favela de Paraisópolis: sua parca arborização e a divisa com parte nobre do bairro Morumbi em São Paulo}%
    \begin{center}%
        \includegraphics[width=1.0\textwidth]{./imagens/paraisopolis.jpg}%
    \end{center}%
    \fonte{Foto da esquerda de Gustavo Roth; foto da direita de Tuca Vieira/Folha Imagens, disponível em: \url{http://www.scielo.br/scielo.php?script=sci_arttext&pid=S0103-49792010000200005} Acesso em 11 de novembro de 2014}
\end{figure}%

Pode-se observar nos três pilares clássicos do desenvolvimento sustentável o papel relevante dos transportes. \citeauthoronline{VASCONCELLOS2012} (\citeyear{VASCONCELLOS2012}) aponta ainda que a energia gasta na mobilidade por habitantes de uma cidade, ou seja, quanta energia os moradores de um município precisam para se deslocarem permite ter uma ideia do ``grau de sustentabilidade'' da mesma. Dessa maneira, cabe uma breve discussão sobre transporte sustentável.

Para Black, um transporte sustentável seria aquele que atende às ``atuais necessidades de transporte e mobilidade e não deve comprometer a capacidade das futuras gerações satisfazerem as suas próprias necessidades'' \cite[p.151]{BLACK1996}, e também que ``provê transporte e mobilidade com combustíveis renováveis, minimizando as emissões prejudiciais ao ambiente local e globalmente, e prevenindo fatalidades, lesões e congestionamentos desnecessários'' \cite[p.12]{BLACK2010}.

Banister (\citeyear{BANISTER2005,BANISTER2008}) aponta algumas medidas a serem perseguidas para que se possa alcançar um transporte sustentável:
\begin{compactitem}[]
\item (i) reduzir a necessidade de viajar;
\item (ii) encorajar a troca para modos de transporte coletivo ou não motorizado;
\item (iii) reduzir o comprimento das viagens;
\item (iv) incentivar a adoção de sistemas e tecnologias de transporte mais eficientes, tanto para carga quanto para passageiros;
\item (v) reduzir a utilização de carros e caminhões de carga nas áreas urbanas;
\item (vi) reduzir, na fonte, ruídos e emissões dos veículos,
\item (vii) incentivar a utilização mais eficiente e ambientalmente consciente do estoque de veículos;
\item (viii) melhorar a segurança de pedestres e de todos usuários das (rodo)vias;
\item (ix) melhorar a atratividade das cidades para seus moradores, trabalhadores, compradores e visitantes.
\end{compactitem}

Embora cientes (organismos internacionais, governos e comunidades científicas) de medidas que corroborariam para o estabelecimento de um transporte sustentável, os padrões de mobilidade observados indicam uma dependência cada vez maior do automóvel (com poucas exceções), seja nos países desenvolvidos \cite{BANISTER2005}, seja nos países em desenvolvimento \cite{VASCONCELLOS2012}. \citeauthoronline{BANISTER2005} (\citeyear{BANISTER2005}) informa que entre 1984 e 1994 houve um aumento de 31\% na posse de veículos e que estimava-se chegar a 50\% em 2020. Ele também indica que a maior parte da frota (70\% em 2005) encontrava-se nos países desenvolvidos.

Entretanto, isso não significa que a posse de carros não esteja crescendo nos países em desenvolvimento. No Brasil, a frota de automóveis vem crescendo desde 1960 conforme pode ser observado na Tabela \ref{tab:venda-veic-br}, sendo que em 2009, cerca de 79\% da frota total de veículos era composta por automóveis \cite{VASCONCELLOS2012}. Esse fato somado ao de que o automóvel é o modo que apresenta o maior consumo energético (ver Tabela \ref{tab:gep-modo}) levam a concluir que o desenvolvimento no Brasil também trilha o caminho da insustentabilidade.

\clearpage
\begin{table}[htb]
    \IBGEtab{%\renewcommand{\arraystretch}{1.5}%%\ABNTEXfontereduzida%
	    \renewcommand{\arraystretch}{1.5}
        \caption{Venda interna de veículos no Brasil entre 1960 e 2009}
		\label{tab:venda-veic-br}
    }{%
	    \begin{tabular}{p{2.00cm} P{4.0cm} P{4.0cm} P{4.0cm}}
            \toprule
	           \headerTabCenterCell{Ano} &
		       \headerCell{Autos} &
		       \headerCell{Total} &
		       \headerCell{Fator de crescimento (total)} \\
		    \midrule \midrule
		        1960&
		        40.980&
		        131.499&
		        1\\
		    \midrule
		        1970&
		        308.024&
		        416.704&
		        3,2\\
		    \midrule
		        1980&
		        739.028&
		        980.261&
		        7,5\\
		    \midrule
		        1990&
		        532.906&
		        712.741&
		        5,4\\
		    \midrule
		        2000&
		        1.176.774&
		        1.489.481&
		        11,3\\
		    \midrule
		        2009&
		        2.474.764&
		        3.141.240&
		        23,9\\
		    \bottomrule
		\end{tabular}
    }{%
		\fonte{Adaptado de \cite[p.29]{VASCONCELLOS2012}}
		}
\end{table}

\begin{table}[htb]
    \IBGEtab{%\renewcommand{\arraystretch}{1.5}%%\ABNTEXfontereduzida%
	    \renewcommand{\arraystretch}{1.5}
        \caption{Consumo energético teórico dos modos de transporte em lotação plena}
		\label{tab:gep-modo}
    }{%
	    \begin{tabular}{p{4.00cm} P{4.0cm}}
            \toprule
	           \headerTabCenterCell{Modo de Transporte} &
		       \headerCell{gramas equivalentes de petróleo para mover um passageiro por um quilômetro}\\
		    \midrule \midrule
		        ônibus comum&
		        4,1\\
		    \midrule
		        metrô&
		        4,3\\
		    \midrule
		        motocicleta&
		        11,0\\
		    \midrule
		        automóvel&
		        19,3\\
		    \bottomrule
		\end{tabular}
    }{%
		\fonte{Adaptado de \cite[p.84]{VASCONCELLOS2012}}
		}
\end{table}

Se parece ilógica e insustentável a adoção do automóvel particular como modo principal de locomoção, por que ele continua tão bem cotado? A resposta a essa pergunta parece ser uma soma de fatores que o leva a ser um ícone, culturalmente simbólico e economicamente valorizado. 
Sobre o caráter simbólico, \citeauthoronline{BANISTER2005} (\citeyear{BANISTER2005},. p.05) expõe que o carro é visto como ``seguro, sempre disponível e nunca muito longe do seu motorista''. \citeauthoronline{URRY2001} (\citeyear{URRY2001}) indica ainda outros fatores que contribuem para esse \emph{status} do carro: 
\begin{compactitem}[]
\item (i) como um objeto manufaturado, nascido com o fordismo, é um ícone do sucesso capitalista; 
\item (ii) depois da moradia, é o principal bem de consumo que confere \emph{status} social ao indivíduo;
\item (iii) é um objeto de suficiente complexidade que sintetiza e ilustra um avanço tecnológico;
\item (iv) confere mobilidade individual e, portanto, liberdade para algumas escolhas como horários de saída e rotas adotadas;
\item (v) é revestido de um discurso pela mídia e pela indústria cultural que o liga ao sucesso e ao progresso.
\end{compactitem}

Sob a perspectiva econômica, na indústria brasileira, o ramo automobilístico tem tido um papel bastante central. Nos anos 1950 foram instaladas as três maiores montadoras à época na região de São Bernardo do Campo (SP), o que gerou emprego, aqueceu a indústria e também estimulou o nascimento de toda uma geração de motoristas de carro. Desde então, há uma pressão crescente por mais vias, maiores, melhores e mais fluidas.
%Em 1990, com a estabilização da economia e o controle da inflação, houve o fortalecimento do setor da construção civil e a popularização do financiamento de motos e carros. Esses elementos geraram uma megalópole com, por exemplo, \emph{shopping centers} que dispõem de gigantesca quantidade de vagas de estacionamento e condomínios com pelo menos uma vaga de garagem por apartamento, sem que tudo isso seja devidamente comportado pelos espaços de circulação.
Dado que o espaço é finito, ao aumentar os espaços de circulação, diminuem-se os espaços disponíveis para abrigarem as atividades das pessoas. Há mais de 50 anos \apudonline[p.63]{OWEN1956}{BLACK2010} já reconhecia que:

\begin{citacao}
O problema do congestionamento se tornou tão grande que muitas comunidades estão chegando à conclusão de que nunca haverá avenidas nem vagas de estacionamento suficientes que permitam o movimento de todas as pessoas em carros particulares.
\end{citacao}

A RMSP sofre das contradições de políticas que apontam para direções diferentes, quando não antagônicas. No âmbito do município de São Paulo, conta-se com a ``Lei de Mudanças Climáticas'' \cite{LEICLIMASP2009} que prevê a redução de 30\% nas emissões dos gases do efeito estufa, além de substituição integral do uso de combustíveis fósseis por renováveis na frota de transporte público. No âmbito estadual, o Plano de Controle de Poluição Veicular 2011-2013 \cite{PCPV2011} indica, entre outros objetivos, a adoção da inspeção ambiental de veículos, uma (única) medida que incide sobre o transporte privado individual. Já no âmbito federal, para garantir aquecimento econômico e minimizar a taxa de desemprego, o Imposto sobre Produtos Industrializados (IPI) dos carros nacionais novos 1.0 foi a zero no primeiro semestre de 2012, sendo que até o final de 2014 não terá retornado ao patamar dos 11\% \cite{FAZENDA2014}.

Em alguma medida o conjunto das políticas públicas transparece um desejo de não restringir a posse do carro, mas seu uso. Ou seja, deseja-se ao mesmo tempo desviar do impacto econômico que uma diminuição das vendas de carros geraria e regular o uso dos automóveis. Essa é a abordagem liberal que diversas cidades, de vários países do mundo vem adotando. Isto é, não se deseja impor restrições legais ou econômicas, mas entender e estimular comportamentos mais interessantes para o conjunto da sociedade e que contribua para a construção de cidades mais sustentáveis. Todavia, \apudonline[p.7]{GILBERT2000}{BANISTER2005} deixa o alerta de que ``há uma ligação entre a posse do carro e uso do carro, e qualquer estratégia coerente para reduzir o uso do carro está fadada ao fracasso se realmente não abordar a causa da mobilidade insustentável, ou seja, o carro''.


\clearpage
\section{Intersecções e Sobreposições}

Não apenas o movimento feminista e o embrião da concepção de gênero datam do final do século XIX; a intersecção entre gênero e mobilidade. Em 1895, \citeauthoronline{WILLARD1895} publicou seu livro \emph{A Wheel whithin a Wheel} em que narra como ela, mulher, aos 53 anos, aprendeu a andar de bicicleta (ver Figura \ref{fig:willard}). Ela não escreveu um livro sobre mobilidade, nem sobre gênero. Porém, ela aborda essas questões a partir dessa sua experiência. Ela toca na questão de gênero, por exemplo, ao falar do vestuário de uma ciclista:

\begin{citacao}
Se as mulheres pedalarem, ao fazê-lo elas devem vestir-se mais racionalmente do que foram acostumadas. E se elas fizerem isso, muitos preconceitos concernentes ao que elas estariam autorizadas a vestir cairão por terra. (Livre tradução de \citeauthoronline{WILLARD1895}, \citeyear{WILLARD1895}, p.39)
\end{citacao}

\begin{figure}[htb]%
    \caption{\label{fig:willard}Frances Willard aprendendo a andar de bicicleta}%
    \begin{center}%
        \includegraphics[width=0.37\textwidth]{./imagens/Willard-p56.jpg}%
    \end{center}%
    \fonte{\cite[p.56]{WILLARD1895}}
\end{figure}%

O que se esperava que uma mulher vestisse àquela época? Como pedalar com vestidos tão longos que sempre cobriam os pés e contavam com muitos babados, plissados, franjas e passamanarias%
\footnote{Essas características da indumentária feminina utilizada no final do século XIX podem ser percebidas ora numa pintura de Cézanne (\emph{Madam Cézanne} num vestido vermelho, de 1888/1890), ora nas peças expostas no Museu da Moda em Canela (\url{http://www.museudamodadecanela.com.br/} - acesso em 15 de novembro de 2014) ou no \emph{The Metropolitan Museum of Art} (\url{http://www.metmuseum.org/toah/hd/wrth/hd_wrth.htm} - acesso em 15 de novembro de 2014).}?
%traje da Princesa Isabel: http://varelanoticias.com.br/vestido-usado-por-princesa-isabel-esta-em-exposicao/
%paperdolls vitorianas: https://casitadepapel.wordpress.com/2012/01/25/paper-dolls-victorianas/
Andar de bicicleta não foi para ela apenas um desafio por conta da habilidade manual requerida e/ou idade que possuía. 
Ao longo da obra ela retrata não apenas que ganhou mobilidade, ela adquiriu auto-confiança e vislumbrou possibilidades que antes não reconhecia, como aspirações relativas a seu crescimento pessoal. 
Ela constatou que a imobilidade física feminina atava-se a outras imobilidades, sociais. 
Assim, transformar a forma de se mover no espaço era (e é) também uma forma de transformar as relações de gênero \cite{HANSON2010}. 
E aqui o termo gênero é pertinente na análise da obra de Willard, e não anacrônico, porque tem no cerne as relações de poder estabelecidas entre indivíduos
\cite{SCOTT1986}, em função do que se entende por ``feminino'' e ``masculino''.
%Ao fim da obra ela revela as motivações que a levaram a isso, entre as quais estão o gosto pela aventura, o empoderamento e o fato de muitas pessoas terem-na dito que não conseguiria com sua idade. 
E embora a preocupação a respeito de sustentabilidade seja recente, extemporânea a \citeauthoronline{WILLARD1895}, vale notar que a liberdade adquirida por ela dá-se ao aprender a andar de bicicleta, um modo de transporte não motorizado, com emissões nulas de gases do efeito estufa, de manutenção pouco custosa e, cada vez mais, símbolo de sustentabilidade.

Os primeiros artigos que abordam explícita e articuladamente questões de gênero e de transporte datam do fim da década de 1970, como o editorial de \citeauthoronline{ROSENBLOOM1978} (\citeyear{ROSENBLOOM1978}), em que ela problematiza como serão distribuídas as atividades e, por conseguinte, as viagens da população frente ao fato de que a proporção de mulheres na força de trabalho vinha aumentando rapidamente%
\footnote{Em 1978, 54\% das mulheres casadas dos Estados Unidos eram assalariadas, mais do que o dobro do que se constatou logo após o fim da Segunda Guerra Mundial \cite{ROSENBLOOM1978}.}. À época havia quem afirmasse que conforme as desigualdades salariais diminuíssem e os papeis sociais se alterassem, as diferenças nos padrões de viagens sumiriam. 
\citeauthoronline{ROSENBLOOM1978} discorda e entre seus argumentos figura o de que as variáveis sócio-econômicas tradicionais pouco explicam as relações de poder e os processos de decisão circunscritos ao ambiente doméstico. Outrossim, identifica-se que nas diversas classes socio-econômicas as mulheres que trabalham ainda continuam sendo as principais responsáveis pelas tarefas domésticas e pelo cuidado com as crianças. Um fato retratado pela autora ilustra o desconhecimento completo do fenômeno por parte dos órgãos oficiais de planejamento de transportes: cientistas contratados pelo \emph{U. S. Department of Transportation} chegaram a declarar que estimular os homens a usar o transporte público e  deixar o carro em casa poderia aumentar o consumo de energia e os níveis de poluição, pois as mulheres usariam o carro fazendo viagens mais curtas, em baixas velocidades e com maior consumo de combustível.

%\hl{>>> talvez para fechar o capítulo} Foi da década de 1980 pra cá que o assunto gênero e transporte passou a reter atenção, crescente, da comunidade científica. Pesquisadores(as) começaram a examinar os padrões de mobilidade com o recorte de gênero considerando que há entre os gêneros assimetria de poder, acesso desigual a recursos materiais e diferenças na escolha modal. São comuns duas abordagens, a primeira que considera principalmente as diferenças de gênero decorrentes do mercado de trabalho \cite{HANSON1985} e a segunda que prioriza as diferenças decorrentes do medo feminino da violência masculina \cite{TRENCH1992,GODDARD2006,LOUKAITOU2008}. O escopo deste trabalho se limita à primeira abordagem, o que não significa que a segunda abordagem seja menos importante - ambas são complementares.

Assim como foi para \citeauthoronline{WILLARD1895}, o ganho de mobilidade pode refletir melhores condições de vida para as mulheres. Não se pode esquecer que na maioria das vezes a viagem é atividade meio e não fim, ou seja, as pessoas precisam de um motivo para fazer uma viagem, querem chegar a algum lugar por um propósito. Caso uma mulher saísse de casa sem propósito, por exemplo no Brasil da década de 1970, seria questionada moralmente. Um homem também poderia ser considerado vagabundo na mesma situação - embora, há de se frisar, o julgamento moral sobre ele seria mais condescendente do que sobre ela. \citeauthoronline{DINCAO2012} (\citeyear{DINCAO2012}, p.235) relata que: 

\begin{citacao}
cronistas, viajantes e historiadores [\ldots] exibem um quadro em que a menina ou a mulher [burguesa] candidata ao casamento é extremamente bem cuidada, é trancafiada nas casas, etc.
\end{citacao}


%>>>>> TRABALHO


Desta forma, uma das maneiras de adquirir liberdade de movimento foi poder ter motivos que não domésticos para sair de casa, o que não se deu - nem se dá - sem resistência, como o editorial de \citeauthoronline{ROSENBLOOM1978} já dava pistas. \apudonline{HANSON1995}{HANSON2010} constatara que as mulheres de Massachusetts, Estados Unidos, tinham dificuldade de encontrar \textbf{trabalho} no final dos anos 1980.
Mais recentemente, 
\citeauthoronline{ELMHIRST2003} (\citeyear{ELMHIRST2003}) apontam que mulheres sequer são consideradas para certos postos de trabalho na Indonésia porque não se supõem que possam estar fora de casa após escurecer. Em algumas vilas indianas, \citeauthoronline{RAJU2005} (\citeyear{RAJU2005}), ao caracterizar um projeto que visa o empoderamento feminino, constata que uma das mudanças mais significativas detectadas foi o fato das mulheres poderem sair sozinhas de casa, isto é, poderem existir \emph{per si} no espaço público.
O ``ganhar a rua'' feminino é fundamentalmente ligado ao acesso ao trabalho, ao aumento da participação feminina na população economicamente ativa \apud{KUNZLER1994}{BEST2005} e ao crescimento da renda individual.  \citeauthoronline{MANDEL2004} (\citeyear{MANDEL2004}) mostrou que mulheres que têm mais liberdade para fazer viagens têm maior renda em Porto Novo, Benin.
Mesmo ao considerar países com menor desigualdade%
\footnote{Para medir a desigualdade é comum utilizar o Índice de Gini, cujos valores são tão mais altos quanto maior for a desigualdade da renda familiar. Estados Unidos apresentam um Índice Gini de 34 (2005); Indonésia, de 39,4 (2005); Benin, de 36,5 (2003); Noruega, de 25 (2008); e Brasil, de 55,3 (2001). Fonte: \url{https://www.cia.gov/library/publications/the-world-factbook/fields/2172.html} Acesso em 29 de novembro de 2014.}, como a Noruega, observar-se-á que as mulheres casadas trabalham em localidades mais próximas das residências e têm menos poder de escolha geográfico do que seus maridos no que tange às oportunidades de trabalho \cite{HJORTHOL2000}.

Não obstante, é preciso ter cuidado e não fazer uma relação identitária automática entre \textbf{mobilidade} e \textbf{empoderamento}.
Alterar padrões de mobilidade pode significar alterar relações de poder já que é de alguma forma requisito da acessibilidade à escola, ao trabalho, a hospitais, às lojas, às áreas de lazer, etc.
Então, maior mobilidade pode significar mais equidade em algumas situações, mas não necessariamente em todas. Isto é, não há uma relação biunívoca em que maior mobilidade leve, sempre, a mais acesso e oportunidades iguais a todas pessoas. Essas nuances podem ser observadas quando se perfaz uma abordagem interseccional, como fizeram \citeauthoronline{LAFFERTY1991} (\citeyear{LAFFERTY1991}), \citeauthoronline{LAFFERTY1992} (\citeyear{LAFFERTY1992}) e \citeauthoronline{CRANE2007} (\citeyear{CRANE2007}). \citeauthoronline{GILBERT1998} (\citeyear{GILBERT1998}) faz essa ponderação e encara como demasiado simplista entender a mobilidade como empoderamento e a imobilidade como sinal de falta de poder; afinal, é preciso considerar a espacialização dessa mobilidade, considerando suas características sociais, culturais e econômicas. Talvez uma pessoa se desloque menos por ter mais acesso a oportunidades num raio próximo do seu lar e este lar assim seja localizado porque essa pessoa desfruta de melhor posição social e econômica.


%>>>>> INTERSECCIONALIDADE


\citeauthoronline{LAFFERTY1991} (\citeyear{LAFFERTY1991}) indicam que maior mobilidade não significa mais poder, visto as condições socioeconômicas de quem precisa fazer longas viagens para trabalhar em postos de baixa remuneração. Em artigo de \citeyear{LAFFERTY1992}, com base em dados referentes ao final dos anos 1980 do norte de Nova Jersei (Estados Unidos), \citeauthoronline{LAFFERTY1992} sustentam a hipótese que de as diferenças de gênero na segmentação do mercado de trabalho têm consequências sobre a distribuição espacial (desigual) das minorias. As autoras conseguem atingir uma \textbf{abordagem interseccional} entre raça, gênero e classe neste trabalho. As mulheres afro-descendentes, latinas e brancas distribuem-se de forma diversa no espaço urbano. As negras contam com tempos de viagem maiores e menor grau de confinamento espacial que as latinas e as brancas. A ``cidade da mulher negra'' é maior, mas nem por isso, melhor: elas têm menos acesso a oportunidades de trabalho em regiões próximas das suas residências e geralmente são empregadas em postos de baixa remuneração. A maioria desses postos, ligados ao setor de serviços e não ao de processos de manufatura, são menos vulneráveis ao desemprego. Os homens negros e latinos e as mulheres latinas são contratados mais frequentemente pelo setor manufatureiro e, portanto, são mais suscetíveis ao desemprego estrutural. Essa menor vulnerabilidade ao desemprego da mulher negra, contudo, não implica que as oportunidades de emprego lhes sejam mais acessíveis - elas precisam ir mais longe para consegui-las. Entre as mulheres, as latinas apresentam viagens relativamente curtas, próximo ao padrão das brancas. Porém, contam com alto grau de confinamento e a pior remuneração de todos os grupos analisados. Destarte, \citeauthoronline{LAFFERTY1992} mostram que analisar somente a classe (pela renda, poder de compra, posse de bens, grau de instrução associado), ou somente o gênero, ou somente a raça/etnia pode levar a conclusões muito triviais para fenômenos que são mais complexos. Em estudo mais recente, também nos Estados Unidos, que parte de bases de dados nacionais de 1985, 1995 e 2005, \citeauthoronline{CRANE2007} (\citeyear{CRANE2007}) observa que os tempos de viagens vêm convergindo quando consideradas as várias raças/etnias do mesmo gênero, em especial para mulheres. Ou seja, mulheres negras, brancas, asiáticas e latinas tendem a apresentar menos diferenças entre si ao longo do tempo. Analogamente, esse fenômeno também ocorre dentro do grupo masculino.


Embora a divisão de trabalho por gênero seja identificada como um fator que influencia a mobilidade, costuma-se ver o trabalho doméstico como uma restrição à participação do mercado de trabalho.
Subestima-se o efeito do arranjo familiar no padrão de atividades e de viagens geradas a partir de demandas domésticas.
Se no mercado de trabalho vem sendo traçado um caminho que tende a diminuir o desequilíbrio de gênero, no trabalho doméstico ainda é a mulher a grande responsável pela sua execução.
Cabe então, analisar as \textbf{viagens cujo motivo não seja o trabalho} e outros aspectos relacionados ao arranjo familiar: como a presença de criança interfere na rotina familiar e como se distribui o uso do automóvel entre os membros, quando este está presente.


%>>>> VIAGENS NÃO-TRABALHO


Em 1997, \citeauthoronline{ROOT1999} (\citeyear{ROOT1999}) indicam que cerca de 50\% das viagens feitas por mulheres, nos Estados Unidos, por motivos pessoais (não trabalho) eram, na realidade, para a família.
\citeauthoronline{VANCE2007} (\citeyear{VANCE2007}) trabalham econometricamente com dados em painel da Alemanha, referente ao período de 1996 a 2003, indagando se o sexo (feminino/masculino) desempenha papel relevante na determinação da probabilidade de utilização do automóvel ou da distância percorrida; e se assim o for, se seria esse papel influenciado por outros atributos socieconômicos do indivíduo ou de seu núcleo familiar. 
Os autores constataram que as mulheres realizam mais viagens do que os homens quando o motivo não é trabalho. E embora o volume de viagens delas seja maior, a relação de dependência do carro para este tipo de viagem é menor - elas utilizam mais outros modos.



%>>>>> PRESENÇA DE CRIANÇA NA FAMÍLIA

\citeauthoronline{VANCE2007} (\citeyear{VANCE2007}) verificaram que situação ocupacional (empregado(a) ou não), número de crianças na família (ver Figura \ref{fig:prob-uso-carro}), facilidade de acesso ao transporte público tiveram influência significativa nas viagens feitas por carro (que não para o trabalho), tanto para homens como para mulheres. Essas mesmas variáveis não influenciaram, porém, a distância média dirigida; apenas o acesso (ou não) ao automóvel incidiu sobre esse efeito. 

\begin{figure}[htb]%
    \caption{\label{fig:prob-uso-carro}Simulação de probabilidade de uso do carro para viagens não-trabalho em função do número de crianças na família, na Alemanha}%
    \begin{center}%
        \includegraphics[width=0.8\textwidth]{./imagens/prob-uso-carro.jpg}%
    \end{center}%
    \fonte{Adaptado de \cite[p.59]{VANCE2007}}
\end{figure}%

A \textbf{presença de criança} é um fator que impacta bastante no padrão de atividades da família, e pode incidir diferentemente sobre pais e mães. Ainda olhando para a Alemanha, o estudo de \citeauthoronline{BEST2005} (\citeyear{BEST2005}) para a região de Cologne indica que mães usam menos frequentemente o carro do que mulheres sem filhos, ao passo que pais usam mais o carro do que homens sem filhos.
Este resultado contradiz o encontrado por \citeauthoronline{VANCE2007}, indicando que é preciso considerar a diversidade dos contextos socioeconômicos e culturais neste tipo de análise e que não pé possível simplesmente transpor conclusões de outros locais para a RMSP.

\citeauthoronline{GODDARD2006} (\citeyear{GODDARD2006}) identificaram que a presença de criança na família causa diferença significativa no comportamento de viagens entre homens e mulheres no norte da Califórnia (Estados Unidos). A presença de criança na família tem grande impacto na quantidade de tempo/distância que a mulher dirige automóvel, mas tal efeito não se observa no comportamento masculino. 
Esse comportamento pode se dever ao fato de que os homens, antes da paternidade, já estão mais familiarizados com o uso do carro e tendem a não trocar sua escolha modal. 
As mulheres, mães, tendem a ser mais pressionadas pelas atividades a serem cumpridas, que não o trabalho. Elas passam a absorver com mais facilidade a viagem de servir passageiro (a criança). Nesta conjuntura vale ressaltar que as escolas, em especial que atendem as crianças nas primeiras idades, geralmente ficam próximas à residência o que estimula que as viagens para levar crianças à escola sejam feitas a pé, pelas mães. Porém, as metodologias utilizadas nos grandes \emph{surveys}, tanto no Brasil como fora, pecam em detectar viagens curtas a pé. Nas Pesquisas OD da RMSP as viagens feitas integralmente a pé, se estiverem num raio inferior a 500 metros da residência, são desconsideradas se não forem com destino ao trabalho ou à escola.



%>>>>> POSSE E USO DO CARRO


\citeauthoronline{FOX1983} já apontava que em \citeyear{FOX1983} as mulheres nos Estados Unidos usavam menos o automóvel e mais o transporte público. Essa priorização masculina no \textbf{uso do automóvel} vem persistindo nos Estados Unidos %\hl{REF}%
e também na Europa. \citeauthoronline{HJORTHOL2000} (\citeyear{HJORTHOL2000}) ao investigar mulheres e homens casados da região de Oslo, Noruega, observou que em famílias que dispunham de um carro, o marido detinha a prioridade do uso. \citeauthoronline{POLK2003} (\citeyear{POLK2003}) indica que os homens usam mais o carro, acumulam mais quilômetros percorridos por ano e fazem mais viagens como ocupantes únicos do que mulheres na Suécia.
Embora ter à disposição um carro para uso privado é o fator que mais influencia o seu uso em viagens motivo ``manutenção do lar'', segundo \citeauthoronline{BEST2005} (\citeyear{BEST2005}), na Alemanha, isso ainda não é suficiente para que as mulheres usem mais o carro do que os homens no geral. 
Estes autores elencam ainda outros fatores que estimulam o uso do carro: renda familiar e participação no mercado de trabalho. Assim, se há mais renda familiar, há mais condições financeiras de adquirir um carro para tê-lo à disposição. E se há mais pessoas na família que fazem parte do mercado de trabalho, há mais interesse em comprar um carro, seja pelo aumento da renda familiar, seja pela necessidade de contar com um modo com flexibilidade de rota.



%>>> SUSTENTABILIDADE


Portanto, as pesquisas apontam para o fato de que, \emph{ceteris paribus}, quando existe a posse do automóvel na família, este fica mais frequentemente à disposição dos homens do que das mulheres.
Dessa forma, as mulheres andam mais de transporte público e, ao andarem de automóvel: (i) são com maior frequência passageiras, remetendo à ideia de ``não andar desacompanhada fora de casa'', e (ii) servem passageiro mais frequentemente, remetendo às tarefas do cuidado \cite{HIRATA2012} com crianças e idosos \cite{ROSENBLOOM2000,ROSENBLOOM2003}.
Se isso por um lado reflete menor autonomia e independência das mulheres por construção histórica, por outro lado, percebe-se também que elas conscientemente indicam estar mais dispostas a uma migração modal para obter um padrão mais sustentável de deslocamentos do que eles. Mulheres e homens demonstram atitudes diferentes em relação à proteção do meio ambiente e à sustentabilidade, apesar da taxa de motorização feminina vir crescendo nos países mais industrializados \cite{ROOT1999}. \citeauthoronline{POLK2003} (\citeyear{POLK2003}) se propõe a estudar se, na Suécia, as mulheres são potencialmente mais adaptáveis a um sistema de transportes mais \textbf{sustentável} do que os homens. A autora conclui que sim, as mulheres tendem a expressar mais preocupação em relação as questões ambientais e declaram maior vontade de reduzir o uso do carro. 
O carro é modo que mais ocupa espaço urbano e com maior consumo energético por pessoa (ver Tabela \ref{tab:gep-modo}); desta maneira, compreender porque alguém faz uso intensivo do carro e também porque um grupo não o utiliza tanto assim, pode abrir caminhos para políticas de transporte mais sustentáveis. 

\begin{table}[htb]
    \IBGEtab{%\renewcommand{\arraystretch}{1.5}%%\ABNTEXfontereduzida%
	    \renewcommand{\arraystretch}{1.5}
        \caption{Consumo energético teórico dos modos de transporte em lotação plena}
		\label{tab:gep-modo}
    }{%
	    \begin{tabular}{p{4.00cm} P{4.0cm}}
            \toprule
	           \headerTabCenterCell{Modo de Transporte} &
		       \headerCell{gramas equivalentes de petróleo para mover um passageiro por um quilômetro}\\
		    \midrule \midrule
		        ônibus comum&
		        4,1\\
		    \midrule
		        metrô&
		        4,3\\
		    \midrule
		        motocicleta&
		        11,0\\
		    \midrule
		        automóvel&
		        19,3\\
		    \bottomrule
		\end{tabular}
    }{%
		\fonte{Adaptado de \cite[p.84]{VASCONCELLOS2012}}
		}
\end{table}

 
%>>DIST, MODO, TEMPO

\citeauthoronline{SCHWANEN2002} (\citeyear{SCHWANEN2002}) indicam que são variáveis importantes de análise a \textbf{distância} viajada, o \textbf{modo} utilizado e o \textbf{tempo} de viagem - este último é um fator que pesa bastante nas decisões associadas à viagem (fazê-la ou não, que modo utilizar e que rota percorrer).

\citeauthoronline{FOX1983} indicava os padrões de viagens das mulheres nos Estados Unidos em \citeyear{FOX1983}: elas faziam menos viagens, mais curtas e rápidas. Quase dez anos depois, \citeauthoronline{IBIPO1992} (\citeyear{IBIPO1992}) revisita a hipótese de que mulheres fazem viagens mais curtas que homens em função de suas socialmente construídas atribuições domésticas. O autor foca-se no tipo de família e toma para essa análise a variável número de trabalhadores(as) da família. Assim, constitui-se o grupo das famílias em que há um(a) trabalhador(a) e daquelas com dois trabalhadores(as). Com dados de Baltimore, Estados Unidos, no grupo em que o arranjo familiar conta com mais de uma pessoa que trabalha, as diferenças entre distâncias de viagens de homens e mulheres tendem a ser maiores - ainda que sejam controlados outros fatores como, por exemplo, presença de criança na família.

Também no EUA, já no século XXI, \citeauthoronline{CRANE2007} (\citeyear{CRANE2007}) utiliza dados de 1985, 1995 e 2005 em suas análises e novamente afirma que mulheres fazem viagens mais curtas que homens. \citeauthoronline{CRANE2007} vai além e contesta estudos que apontam que o \emph{gender gap} dos tempos de viagem esteja diminuindo e das distâncias de viagem tenham até sumido em algumas áreas. Constata que ainda persistem diferenças: as distâncias percorridas por homens e por mulheres convergem muito lentamente e os tempos divergem.

Na Europa, o panorama se mantém semelhante ao dos Estados Unidos. \citeauthoronline{FAGNANI1983} (\citeyear{FAGNANI1983}) estuda os padrões de deslocamento das mulheres em Paris e constata que lá também elas  realizam viagens mais curtas.
\citeauthoronline{SCHWANEN2002} (\citeyear{SCHWANEN2002}), a partir do \emph{National Travel Survey} holandês de 1998, analisam tempo, motivo e modo da viagem. Encontram evidências que fatores sociodemográficos (sexo, número de trabalhadores(as) na família, idade e grau de instrução) e contexto espacial da localização da residência influenciam o tempo médio de viagem diário. Os efeitos da posse de carro e da renda se dão de forma indireta e interagem com a escolha modal ou o número de trabalhadores(as) da família, segundo os autores. Do ponto de vista da interação com características urbanas, os tempos de viagem de carro tendem a aumentar quanto maior for o grau de urbanização e quanto mais policêntrica for uma região.

No conjunto Ásia e Oceania, o diagnóstico permanece.
Ao analisar as viagens motivo trabalho da região metropolitana de Melbourne, Austrália, \citeauthoronline{HOWE1982} (\citeyear{HOWE1982}) expõem que as taxas de participação feminina na força de trabalho australiana eram baixas à época, levanta a hipótese de que isso se dê por alguma dificuldade de acesso às oportunidades de trabalho. Constata que mulheres fazem viagens mais curtas que homens e conclui que para incrementar a participação feminina no mercado de trabalho é preciso que ofertas de emprego estejam mais distribuídas espacialmente. 
\apudonline{CUSSET1997}{VASCONCELLOS2001} constata que as mulheres em Hanoi (Vietnam) fazem a maior parte das viagens diárias a pé (54\%); para os homens essa taxa cai para cerca de 39\%, indicando maior motorização deles.
\citeauthoronline{SONG2003} (\citeyear{SONG2003}) também constata que para Seul, Coreia, tanto os tempos quanto as as distâncias de viagens são menores paras as trabalhadoras. Nessa cidade, trabalhadores(as) solteiros(as) contam com distâncias de viagem maiores que os(as) casados(as) e, merece destaque o detalhe de que trabalhadores cujas esposas trabalham têm viagens mais curtas do que aqueles cujas esposas não trabalham. Por fim, \citeauthoronline{SONG2003} aponta que a responsabilidade por cuidar das crianças é um fator de encurtamento das viagens das mulheres casadas coreanas.


No Brasil, em específico na RMSP, \citeauthoronline{VASCONCELLOS2001} (\citeyear{VASCONCELLOS2001}) indica também que as mulheres fazem menos viagens e andam mais a pé do que os homens. Como os padrões de deslocamentos das mulheres são muito moldados pelas atividades ligadas ao ambiente doméstico, \citeauthoronline{STRAMBI1998} (\citeyear{STRAMBI1998}), com base na Pesquisa OD-1987, lançam olhar sobre o núcleo familiar e constatam que, 
para categorias de menor renda, o número médio de viagens cai conforme cresce o tamanho da família, o que pode indicar insuficiência de renda para arcar com as tarifas de transporte dos ``membros adicionais''. Percebe-se que o estudo da mulher, da \textbf{família} e da mulher na família são relevantes, pois são fortes fatores sociais e culturais condicionantes de comportamentos. \citeauthoronline{VASCONCELLOS2001} (\citeyear{VASCONCELLOS2001}, p.119) endossa essa abordagem ao dizer que:

\begin{citacao}
O papel das mulheres é especialmente relevante para entender os padrões diários de deslocamento nas famílias dos países em desenvolvimento. A melhor forma de entender estes padrões é começando pela divisão de tarefas domésticas e depois examinar condicionantes culturais e religiosos de sua mobilidade.
\end{citacao}



% >>CADEIA DE VIAGENS


\citeauthoronline{MCGUCKIN1995} (\citeyear{MCGUCKIN1995}), com base no  \emph{Nationwide Personal Transportation Survey} de 1995 estadunidense examinam o \textbf{encadeamento de viagens} feitas durante a semana de homens e mulheres em fase adulta. Constatam que mulheres fazem cadeias de viagens com mais segmentos para que possam acomodar suas responsabilidades domésticas. Essa segmentação é ainda mais sentida por elas quando se conta com a presença de criança na família. As autoras ainda apontam uma mudança na dinâmica doméstica em relação às tarefas e responsabilidades.
Dez anos depois, \citeauthoronline{MCNUCKIN2005} (\citeyear{MCNUCKIN2005}) publicaram estudo sobre o encadeamento de viagens com início na residência e término no trabalho de acordo com sexo (feminino/masculino) e ciclo de vida a partir de base de dados nacionais dos Estados Unidos de 2001. Constataram que em lares com pai e mãe, em que ambos trabalham, as viagens cujo propósito de servir passageiro é deixar as crianças na escola têm maior probabilidade de serem acomodadas na cadeia de viagens da mulher do que do homem.
Na Europa, situação semelhante é encontrada. Na Espanha e no Reino Unido, em 1997, grande parte das viagens femininas era para servir passageiro, em sua maioria, filhos(as) \cite{ROOT1999}. 
Em Oslo, mulheres casadas fazem mais encadeamento de viagens por conta de responsabilidades domésticas do que homens casados \cite{HJORTHOL2000}.
Logo, se houve algum rearranjo na divisão de tarefas domésticas, essa mudança não foi profunda o suficiente para gerar alterações perceptíveis nos padrões de deslocamentos de homens e mulheres em diversos países.
%\cite{DALMASO2009}

%>>>GÊNERO E MOBIliDADE & VICE-VESA

Como o gênero configura e influencia a mobilidade, e como a mobilidade configura e influencia o gênero é o foco da discussão empreendida por \citeauthoronline{HANSON2010} (\citeyear{HANSON2010}). 
Entretanto, o mais comum na pesquisa de gênero e mobilidade é um campo de conhecimento contemplar o outro, com pouco esforço de buscar literatura ou metodologia de intersecção entre as áreas e, com isso, parte-se de conceitos e hipóteses que conduzem a resultados pouco convergentes ao final. \citeauthoronline{HANSON2010} (\citeyear{HANSON2010}, p.09) afirma ainda que em geral o debate ocorre em termos muito genéricos em torno da mobilidade/imobilidade com reflexos de uma ideologia dual que identificam a mulher/feminilidade com a casa, o espaço doméstico, movimentos restritos e o homem/masculinidade com a rua, o espaço público, movimentos livres. 
Enxergar o mundo a partir de um código binário de gênero foi uma construção longa e aboli-la por completo é uma tarefa a que este trabalho não se propõe, embora sempre que possível tente considerar a categoria de gênero para além do sexo, binário.

A maior parte da literatura revisada diz respeito aos países do hemisfério Norte, mais industrializados, concentrando-se na Europa e nos Estados Unidos. Isso deve-se principalmente a dois fatores: 
(i) são países em que os temas gênero, transporte e sustentabilidade são mais sistematicamente investigados; 
(ii) alguns países da Ásia também têm alguma produção dentro da temática desta pesquisa, mas muitas vezes em língua que não é de grande difusão no ocidente (como mandarim ou coreano). 
O que realmente chama a atenção é a parca literatura sobre a situação na América Latina. 
A maioria dos estudos encontrados baseiam-se em grandes bases de dados quantitativos que são tratados de forma agregada. 
Daí, conclui-se que, agregadamente, os comportamentos de homens e de mulheres são diferentes, em diversos países e continentes. 
Uma série de aspectos relacionados à viagem são analisados para fazer essa constatação: quantidade por pessoa, tempo, distância, encadeamento, modo e motivo. Outros aspectos, ligados às características de indivíduos, são frequentemente envolvidos nas análises: sexo, idade, estágio do ciclo de vida (individual e familiar), estado civil, presença de filhos na família, papel exercido dentro da família, situação ocupacional e renda.


%As grandes compras geralmente são feitas utilizando-se o carro, especialmente num contexto de padrão de consumo em que se expandem os grandes hipermercados, no Brasil, que precisam de grandes áreas urbanas e vendem em grandes volumes. Desta forma, estes empreendimentos têm localização menos central e buscam estar próximos de grandes avenidas. Essas condições de contorno (facilidade de acesso por carro e estímulo a grande volume de compras) tendem a levar o usuário a preferir o carro como meio de transporte.

%Já as pequenas compras geralmente são feitas a pé e estas são majoritariamente feitas pelas mulheres. Trata-se da compra diária na padaria, na farmácia, no mercado do bairro, entre outras. Porém, no Brasil, as pesquisas Origem-Destino costumam ignorar as viagens feitas a pé num raio inferior a 500m, o que inclui grande parte das viagens de “manutenção” como as de pequenas compras de abastecimento doméstico (Metrô, 2007).


%No Brasil, o olhar integrador entre transportes, planejamento urbano, meio ambiente e aspectos sociais tem sido cada vez mais frequente. Dois exemplos são o Estatuto da Cidade (\citeyear{ESTATUTOCIDADE}), obrigatório para cidade com mais de 20 mil habitantes, e o Plano Nacional de Mobilidade Urbana (\citeyear{PNMU}), obrigatório para cidades com mais de 500 mil habitantes. Tratam-se de dois instrumentos legais que norteiam elaboração de políticas públicas e, de acordo com \citeauthoronline{IEMA2010}(\citeyear{IEMA2010}):

%\begin{citacao}
%Estatuto da Cidade estabelece o direito às cidades sustentáveis para a atual e as futuras gerações, [sendo esse direito] compreendido como o acesso ao solo urbano, moradia, saneamento, infraestrturua, trabalho, lazer e serviços públicos.
%\end{citacao}

%\begin{citacao}
%A descrição e análise dos fenômenos de megalopolização que ocorrem durante os últimos 500 anos surpreendem pela convergência de padrões na maioria das megalópoles latino-americanas aqui apresentadas. Eles não podem ser atribuídos à história, mas apontam para forças macroestruturais que promovem um desenvolvimento urbano que converge para a ``insustentabilidade'' das megalópoles na era da globalização. \cite{FREITAG2007}
%\end{citacao}

%All else being equal, the average commuting time and the modal split in European cities are more strongly associated with the distribution of employment and population across the urban area and with urban siz \cite{SCHWANEN2002a}
