% ---
% Capitulo Métodos
% ---
\chapter{Considerações Metodológicas}\label{chap:metodo}

%constructo como sendo uma definição mental de uma ideia de pesquisa, estabelecida com base na teoria subjacente e/ou na experiência e na intuição do pesquisador

Esta pesquisa lida fundamentalmente com dados secundários, principalmente do Metrô-SP através de suas Pesquisas OD e, em menor grau, com dados socioeconômicos advindos de pesquisas PNAD%
\footnote{A PNAD é a Pesquisa Nacional por Amostra de Domicílios, realizada bianualmente pelo IBGE (referência no mês de setembro) com objetivo de investigar características socioeconômicas da população. Fonte:\url{http://www.previdencia.gov.br/arquivos/office/3_081014-105206-595.pdf} Acesso em 20 de novembro de 2014} e censos do IBGE%
\footnote{O censo demográfico é uma pesquisa realizada decenalmente pelo IBGE com objetivo de caracterizar sociodemograficamente a população brasileira. O primeiro foi feito em 1872 e o último data de 2010. Fonte:\url{http://cod.ibge.gov.br/234lq} Acesso em 20 de novembro de 2014}.
Por dados secundários entendem-se aqueles que ``já foram coletados para objetivos que não os do problema'' \cite[p.127]{MALHORTA2001}. Trabalhar com dados secundários traz vantagens e desvantagens. Como vantagens tem-se o acesso relativamente fácil, o baixo custo de coleta e a rapidez de obtenção dos dados. 
Como desvantagem tem-se que o propósito da coleta dos dados difere daquele para que estão sendo utilizados aqui. Isto é, a formulação dos questionários e o desenho dos bancos de dados buscam responder a perguntas diferentes das propostas por esta dissertação.
Essa desvantagem não inviabiliza o uso dos dados, mas é uma informação que deve estar em mente ao manipulá-los e analisá-los. \citeauthoronline{MALHORTA2001} (\citeyear{MALHORTA2001}, p.128) indica que os dados secundários podem auxiliar a:
(i) identificar o problema; 
(ii) definir melhor o problema; 
(iii) desenvolver uma abordagem do problema; 
(iv) formular uma concepção de pesquisa adequada; 
(v) responder a certas perguntas de pesquisa e testar algumas hipóteses; 
(vi) interpretar dados primários com mais critério.

Nesta dissertação os bancos de dados e manuais de referência das respectivas Pequisas OD foram requisitados por meio de formulário \emph{online} do e-SIC
\footnote{O e-SIC é o Sistema Eletrônico do Serviço de Informações ao Cidadão que possibilita que qualquer pessoa, física ou jurídica, encaminhe pedidos de acesso à informação, acompanhe o prazo e receba a resposta da solicitação realizada para órgãos e entidades governamentais. Fonte: \url{http://www.sic.sp.gov.br/} Acesso em 20 de abril de 2014} do governo do Estado de São Paulo e disponibilizados pelo Metrô-SP em mídia digital para retirada em dentro do prazo de 20 dias estabelecido pela lei de acesso à informação pública.
Os dados provenientes de PNAD, censos ou outras pesquisas foram obtidos por meio de relatórios públicos, disponibilizados \emph{online} e têm suas fontes indicadas ao longo do texto. 
Entende-se que os dados advindos das Pesquisas OD podem auxiliar na tarefa de identificar se existem diferentes padrões de mobilidade de acordo com o gênero na RMSP, pois contêm informações sobre os deslocamentos de indivíduos, com representatividade em suas respectivas zonas e indicados os respectivos fatores de expansão nos bancos de dados. 
Para investigar a evolução e os motivos dessas diferenças têm-se como base as hipóteses advindas da \textbf{revisão da literatura}: ao gênero feminino é atribuído um determinado papel social e lhe é inputada a realização de uma maior diversidade de trabalhos (remunerados e não remunerados) que precisam ser acomodados no cotidiano. 
Os dados socieconômicos podem dar indícios daquelas hipóteses ligadas à influência das características socioeconômicas sobre as ações de indivíduos. 
Dados como idade e ocupação (se trabalha, se estuda, etc.) podem dar indícios para compreender a influência do ciclo de vida no comportamento dos indivíduos. Dados como situação na família (pessoa responsável, cônjuge, etc.) e presença de filhos podem auxiliar a compreender a dinâmica familiar e seu reflexo nos padrões de atividades das pessoas. 
Dados como posse de automóveis, motocicletas, bicicletas e modos utilizados nas viagens, podem contribuir para compreender a relação que é estabelecida com o transporte público e o transporte privado, mais especificamente o carro. 

Os dados secundários das \textbf{Pesquisas OD} foram analisados segundo alguns critérios, observados no Quadro \ref{qua:dados-sec}. 
Os objetivos primários da coleta dos dados são apresentados na Seção \ref{sec:period-obj}. 
Em relação à natureza dos dados, para que fossem melhor utilizados nesta dissertação, foram feitas compatibilizações entre as zonas (geográficas) de análise e os diversos bancos de dados, processo mais detalhadamente explicado no Capítulo \ref{chap:trat-dados}. 
Em relação à confiabilidade, os dados foram adquiridos diretamente do Metrô-SP, responsável pela coordenação da coleta e compilação dos dados, e que goza de boa reputação e pioneirismo no Brasil na realização de pesquisas dessa natureza. 
Em relação à atualidade dos dados, no Capítulo \ref{chap:pesquisa-od} esclarece-se a periodicidade e datas de referência das OD-1977, OD-1987, OD-1997 e OD-2007. 
Em relação às especificações e metodologia das Pesquisas OD, no Capítulo \ref{chap:pesquisa-od} são descritos brevemente os métodos de coleta, amostragem, conceitos utilizados, além das decorrentes limitações.

\begin{quadro}[htb]
    \IBGEtab{
        \renewcommand{\arraystretch}{1.5}
        \ABNTEXfontereduzida
        \caption[Critérios para Avaliação de Dados Secundários]{\label{qua:dados-sec}Critérios para Avaliação de Dados Secundários}
	}{%
        \begin{tabular}{P{5.0cm}P{10.00cm}}
           \toprule
		       \headerCell{Critérios} & 
		       \headerCell{Questões} \\
		    \midrule\midrule
		        Objetivo&
		        Por que os dados foram coletados?\\
		    \hline
		        Natureza&
		        Definição de variáveis chave; unidades de medição; categorias usadas e relações examinadas\\
		    \hline
		        Confiabilidade&
		        Experiência; credibilidade; reputação e integridade da fonte\\
		    \hline
		        Atualidade&
		        Prazo entre coleta e publicação; frequência das atualizações\\
			\hline
		        Especificações e Metodologia&
		        Método de coleta de dados; índice de respostas; qualidade dos dados; técnica de amostragem; tamanho da amostra; criação do questionário e trabalho de campo\\
			\bottomrule
		\end{tabular}
	}{%
		\fonte{Adaptado de \cite[p.129]{MALHORTA2001}}
    }
\end{quadro}

Numa análise dos padrões de viagem do sudeste de Michigan, entre 1965 e 1980, \citeonline{KITAMURA1994}  concluíram que o padrão de deslocamentos diários não é estável ao longo do tempo. Também nos Estados Unidos, \citeonline{MCNUCKIN2005} utilizam dois \textit{surveys} nacionais para avaliar tendências nos encadeamentos de viagem. 
\citeonline{LOTTA2011} utilizam dados (intermitentes) de quase 30 anos de \textit{surveys} nacionais para avaliar tendências na mobilidade das pessoas na Suécia, focando no gênero e nos coortes.
\citeonline{CADESTIN2013} usam o \textit{National Travel Survey} de 1973 a 2007 para analisar diferenças de mobilidade entre mulheres e homens com mais de 55 anos na França.
Deste modo, além de estar entre os objetivos desta dissertação avaliar \textbf{tendências ao longo do tempo}, diversos autores indicam ser relevante elaborar estudos longitudinais, inclusive com o foco de gênero. Para isso ser possível, foi necessário criar um Banco de Dados Unificado (BDU).
Para a organização do BDU foi elaborado um \textit{layout} unificador das variáveis e, para cada variável, foi  necessário compatibilizar categorias e fazer testes de validação. Essas transformações foram feitas utilizando linguagem \textit{python} e as rotinas estão disponíveis no Anexo \ref{chap:anexo_rotinas}.
Também foram feitos testes de validação e de consistência globais do BDU e, por conseguinte, excluídos alguns registros. 

É possível fazer análises agregadas (por zonas) ou desagregadas (por famílias ou indivíduos).
As \textbf{análises desagregadas} permitem, por exemplo, avaliar o comportamento da demanda e considerar modelos baseados em teorias de atividades humanas, que consideram condicionantes familiares e externos aos deslocamentos ao invés das viagens isoladamente \cite{JONES1981}.
A desagregação permite considerar como unidade de estudo a família ou o indivíduo, e tem sido indicada como vantajosa do ponto de vista conceitual \cite{ORTUZAR1994}. 
Para ser possível, neste trabalho, elaborar análises desagregadas, novas variáveis de viagens, pessoas e famílias foram criadas.
A descrição mais detalhada de todo este \textbf{tratamento de dados} está no Capítulo \ref{chap:trat-dados}. 

O Capítulo \ref{chap:analises} contém análises, resultados e discussões - realizados utilizando o software livre de estatística computacional R\footnote{Página do projeto R: \url{https://www.r-project.org/}}.
Com objetivo de obter alguma familiaridade que essa grande massa de dados que representa o BDU, foram realizadas algumas \textbf{estatísticas descritivas} (tabelas de frequências, box plot, histogramas, distribuições, entre outras), segmentadas por ano. Essas análises, apresentadas na Seção \ref{sec:bd-estat-descr}, permitem perceber macro tendências de algumas variáveis no tempo.
Na sequência, na Seção \ref{sec:analises-preliminares} são apresentadas algumas \textbf{análises preliminares} dos dados, feitas simultaneamente para os bancos de dados das quatro \emph{cross-sections} (1977, 1987, 1997 e 2007). 
Conforme indicado pela literatura, buscou-se compreender o comportamento de pessoas e famílias, ao longo do tempo, segundo modo, motivo, duração, distância e número de viagens. Aqui, já se começa a fazer algumas segmentações das análises pela variável sexo.

A análise das atividades humanas por segmentos da população possibilita identificar necessidades distintas de transporte e, a partir dos padrões de atividades e viagens, tais informações podem ser aplicadas na formulação ou revisão de modelos de projeção de demanda \cite{MAHMASSANI1988}.
Sendo necessário considerar variáveis e categorias relevantes, bem como as interações entre elas, \citeauthoronline{STRAMBI1998} (\citeyear{STRAMBI1998}) utilizaram técnicas de \textbf{segmentação} para identificar grupos razoavelmente homogêneos, viabilizando o estudo do comportamento desses grupos e também dos fatores que diferenciam os grupos.
Neste trabalho, a técnica inicialmente utilizada para a segmentação dos grupos foi a \textbf{análise de conglomerados} (\textit{clusters}).
Como entrada de dados foi utilizados um conjunto de variáveis com atributos de viagem da família. 
Quais foram essas variáveis e quais os critérios utilizados para definição do número de grupos podem ser observados na Seção \ref{sec:analises-clusters}.
Como medida de similaridade foi usada a distância euclidiana e como métodos de agrupamentos foram usados Ward e centroide, sendo este último preferido devido à maior robustez a \textit{outliers}.

Realizada uma nova análise de conglomerados a partir dos grupos formados na primeira, foram investigadas as características das famílias e indivíduos dos grupos formados segundo a similaridade de seus padrões de deslocamento.
Tal investigação foi empreendida de duas formas. 
A primeira delas observou as diferenças percentuais entre valores mínimo e máximo de cada variável (quantitativa) ou de cada categoria de variável (qualitativa).
Aqui visou-se obter um potencial conjunto de variáveis explicativas para regressões que desenvolvidas mais à frente.
A outra forma empregou \textbf{regressões logísticas multinomiais} (Seção \ref{sec:analises-reg-log}) com as mesmas variáveis que foram \textit{input} para a clusterização, visando analisar o peso delas na explicação dos grupos para, daí, eleger uma variável dependente de relevância para explorar melhor na sequência.

Com os resultados da análise de conglomerados e das regressões logísticas foi elencado um conjunto de variáveis explicativas e dependentes.
Considerando a revisão de literatura e teoria subjacente, foram escolhidas quais as variáveis a entrar no modelo de regressão e também, quais os grupos alvo de análise.
Qual o tipo de regressão utilizada foi função da natureza da variável escolhida.
Como a variável dependente escolhida foi TOT_VIAG, um dado de contagem com poucos valores e variância bastante diferente da média, a técnica escolhida foi a \textbf{regressão \textit{quasi-poisson}}, cujos detalhes de aplicação estão na Seção \ref{sec:analises-reg-pisson}.

Por fim, esta pesquisa utiliza um banco de dados secundário em que a informação relativa a gênero nasce quase que exclusivamente da variável sexo, codificada binariamente - certamente uma \textbf{limitação} - muito embora seja o tipo de pergunta que o(a) entrevistador não faz diretamente, mas marca no papel a partir de julgamento visual. 
Isso significa que uma pessoa transsexual cuja performatividade seja feminina, provavelmente será identificada como mulher no questionário. 
Por um lado, este trabalho não traz elemento metodológico inovador que rompa com a leitura binária de gênero, por conta da limitação que os dados de origem impõem. 
Por outro lado, a articulação da variável sexo com outras (como situação familiar), pode apontar na direção de finalmente compreender o gênero para além do sexo na área dos transportes.
